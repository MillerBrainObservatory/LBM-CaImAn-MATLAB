%% Generated by Sphinx.
\def\sphinxdocclass{report}
\documentclass[letterpaper,10pt,english]{sphinxmanual}
\ifdefined\pdfpxdimen
   \let\sphinxpxdimen\pdfpxdimen\else\newdimen\sphinxpxdimen
\fi \sphinxpxdimen=.75bp\relax
\ifdefined\pdfimageresolution
    \pdfimageresolution= \numexpr \dimexpr1in\relax/\sphinxpxdimen\relax
\fi
%% let collapsible pdf bookmarks panel have high depth per default
\PassOptionsToPackage{bookmarksdepth=5}{hyperref}

\PassOptionsToPackage{booktabs}{sphinx}
\PassOptionsToPackage{colorrows}{sphinx}

\PassOptionsToPackage{warn}{textcomp}
\usepackage[utf8]{inputenc}
\ifdefined\DeclareUnicodeCharacter
% support both utf8 and utf8x syntaxes
  \ifdefined\DeclareUnicodeCharacterAsOptional
    \def\sphinxDUC#1{\DeclareUnicodeCharacter{"#1}}
  \else
    \let\sphinxDUC\DeclareUnicodeCharacter
  \fi
  \sphinxDUC{00A0}{\nobreakspace}
  \sphinxDUC{2500}{\sphinxunichar{2500}}
  \sphinxDUC{2502}{\sphinxunichar{2502}}
  \sphinxDUC{2514}{\sphinxunichar{2514}}
  \sphinxDUC{251C}{\sphinxunichar{251C}}
  \sphinxDUC{2572}{\textbackslash}
\fi
\usepackage{cmap}
\usepackage[T1]{fontenc}
\usepackage{amsmath,amssymb,amstext}
\usepackage{babel}



\usepackage{tgtermes}
\usepackage{tgheros}
\renewcommand{\ttdefault}{txtt}



\usepackage[Bjarne]{fncychap}
\usepackage{sphinx}

\fvset{fontsize=auto}
\usepackage{geometry}


% Include hyperref last.
\usepackage{hyperref}
% Fix anchor placement for figures with captions.
\usepackage{hypcap}% it must be loaded after hyperref.
% Set up styles of URL: it should be placed after hyperref.
\urlstyle{same}

\addto\captionsenglish{\renewcommand{\contentsname}{Get started:}}

\usepackage{sphinxmessages}
\setcounter{tocdepth}{1}



\title{caiman\_matlab}
\date{May 23, 2024}
\release{}
\author{unknown}
\newcommand{\sphinxlogo}{\vbox{}}
\renewcommand{\releasename}{}
\makeindex
\begin{document}

\ifdefined\shorthandoff
  \ifnum\catcode`\=\string=\active\shorthandoff{=}\fi
  \ifnum\catcode`\"=\active\shorthandoff{"}\fi
\fi

\pagestyle{empty}
\sphinxmaketitle
\pagestyle{plain}
\sphinxtableofcontents
\pagestyle{normal}
\phantomsection\label{\detokenize{index::doc}}


\noindent{\hspace*{\fill}\sphinxincludegraphics{{overlays1}.png}\hspace*{\fill}}

\sphinxAtStartPar
A pipeline for processing light beads microscopy (LBM) datasets.

\sphinxAtStartPar
For background, theory and design of LBM technology, see the reference \sphinxhref{https://www.nature.com/articles/s41592-021-01239-8/}{publication}.

\sphinxAtStartPar
Currently, inputs to this pipeline are limited to \sphinxhref{https://www.mbfbioscience.com/products/scanimage/}{ScanImage} tiff files.


\chapter{Requirements}
\label{\detokenize{index:requirements}}\begin{itemize}
\item {} 
\sphinxAtStartPar
MATLAB (Tested on 2023a, 2023b, 2024b)

\item {} \begin{description}
\sphinxlineitem{Toolboxes:}\begin{itemize}
\item {} 
\sphinxAtStartPar
Parallel Computing Toolbox

\item {} 
\sphinxAtStartPar
Statistics and Machine Learning Toolbox

\item {} 
\sphinxAtStartPar
Image Processing Toolbox

\end{itemize}

\end{description}

\end{itemize}


\chapter{Algorithms}
\label{\detokenize{index:algorithms}}
\sphinxAtStartPar
The following algorithms perform the main computations and are included by default in the pipeline:
\begin{itemize}
\item {} 
\sphinxAtStartPar
\sphinxhref{https://github.com/simonsfoundation/NoRMCorre}{CNMF} segmentation and neuronal source extraction.

\item {} 
\sphinxAtStartPar
\sphinxhref{https://github.com/flatironinstitute/NoRMCorre/}{NoRMCorre} piecewise rigid motion correction.

\item {} 
\sphinxAtStartPar
\sphinxhref{https://github.com/epnev/constrained-foopsi/}{constrained\sphinxhyphen{}foopsi} constrained deconvolution spike inference.

\end{itemize}


\chapter{Quickstart}
\label{\detokenize{index:quickstart}}
\sphinxAtStartPar
There are 4 steps that require user input:
\begin{enumerate}
\sphinxsetlistlabels{\arabic}{enumi}{enumii}{}{.}%
\item {} 
\sphinxAtStartPar
Convert ScanImage .Tiff to 4D {[}Y, X, Z, T{]} array.

\item {} 
\sphinxAtStartPar
Piecewise\sphinxhyphen{}rigid motion correction.

\item {} 
\sphinxAtStartPar
Plane\sphinxhyphen{}by\sphinxhyphen{}plane 2D neuronal segmentation and deconvolution.

\item {} 
\sphinxAtStartPar
Axial (Z) offset correction.

\end{enumerate}

\sphinxstepscope


\section{First Steps}
\label{\detokenize{get_started/index:first-steps}}\label{\detokenize{get_started/index::doc}}
\sphinxAtStartPar
This section covers the initial steps to perform before running the pipeline to ensure the installation goes smoothly.

\sphinxstepscope


\subsection{Installation}
\label{\detokenize{get_started/install:installation}}\label{\detokenize{get_started/install::doc}}

\subsubsection{Install Recommendation}
\label{\detokenize{get_started/install:install-recommendation}}
\sphinxAtStartPar
The easiest way to download the source code is to visit the \sphinxhref{https://github.com/ru-rbo/caiman\_matlab.git}{github repository}.

\sphinxAtStartPar
The most reliable installation method is to clone/download the repository and move it into a folder on your \sphinxtitleref{userpath}:

\begin{sphinxVerbatim}[commandchars=\\\{\}]
\PYG{o}{\PYGZgt{}}\PYG{o}{\PYGZgt{}}\PYG{+w}{ }\PYG{n+nb}{userpath}

\PYG{n+nb}{ans}\PYG{+w}{ }\PYG{p}{=}
\PYG{+w}{    }\PYG{l+s}{\PYGZsq{}}\PYG{l+s}{/home/\PYGZlt{}username\PYGZgt{}/Documents/MATLAB\PYGZsq{}}
\end{sphinxVerbatim}

\sphinxAtStartPar
Additionally, you can create a \sphinxhref{https://www.mathworks.com/help/matlab/matlab\_env/matlab-startup-folder.html}{startup.m} file located in this same \sphinxtitleref{userpath} directory: \sphinxtitleref{\textasciitilde{}/Documents/MATLAB/startup.m} and add the following code snippet:

\begin{sphinxVerbatim}[commandchars=\\\{\}]
\PYG{c}{\PYGZpc{} \PYGZlt{}HOME\PYGZgt{}/Documents/MATLAB/startum.m}
\PYG{c}{\PYGZpc{} note \PYGZdq{}fullfile\PYGZdq{} isnt needed, but helpfully provides directory autocompletion}
\PYG{n+nb}{addpath}\PYG{p}{(}\PYG{n+nb}{genpath}\PYG{p}{(}\PYG{n+nb}{fullfile}\PYG{p}{(}\PYG{l+s}{\PYGZdq{}path/to/caiman\PYGZus{}matlab\PYGZdq{}}\PYG{p}{)}\PYG{p}{)}\PYG{p}{)}
\end{sphinxVerbatim}

\sphinxAtStartPar
You can tell if the pipeline is added successfully to the path by looking at the file window:

\noindent\sphinxincludegraphics[width=200\sphinxpxdimen]{{matlab_path_explorer}.png}

\sphinxAtStartPar
Notice \sphinxtitleref{/core} and \sphinxtitleref{/packages} are both bright in the files window, this indicates those folders are properly in the MATLAB path but does \sphinxstylestrong{not} mean that within that directory, all subdirectories are also on the path. These two folders contain all of the code the pipeline needs to run and are the only two foldrs that \sphinxstylestrong{must} be on the path.


\subsubsection{By Operating System}
\label{\detokenize{get_started/install:by-operating-system}}
\sphinxAtStartPar
Modern versions of matlab (2017+) solve most Linux/Windows filesystem conflicts. Installation is similar independent of OS.


\paragraph{Windows}
\label{\detokenize{get_started/install:windows}}
\sphinxAtStartPar
The easiest method to download this repository with git is via \sphinxhref{https://gitforwindows.org/}{mysys}
Or just download the code from code/Download.zip above and unzip to a directory of your choosing.

\begin{sphinxadmonition}{note}{Note:}
\sphinxAtStartPar
If you have MATLAB installed on Windows, you won’t be able to run commands from within WSL (i.e. //wsl.localhost/)
due to the separate filesystems. Pay attention to which environment you install.
\end{sphinxadmonition}


\subsubsection{WSL2 (Windows Subsystem for Linux)}
\label{\detokenize{get_started/install:wsl2-windows-subsystem-for-linux}}
\sphinxAtStartPar
Windows subsystem for Linux (WSL/WSL2) is a local environment on your windows machine that is capable of running linux commands using a separate filesystem. As of 2024, Mathworks does not officially support and is not planning support for MATLAB on WSL or WSL2.
If you have MATLAB installed on Windows and wish to use this repository from a WSL instance, see \sphinxhref{https://ww2.mathworks.cn/matlabcentral/answers/1597219-can-microsoft-s-wslg-windows-subsystem-for-linux-gui-support-running-matlab}{this discussion on how to accomplish this}.
This means you will not be able to run matlab from the WSL filesystem (i.e. \sphinxtitleref{//wsl.localhost/}), but you can use a mounted \sphinxtitleref{C://} drive path like so:

\begin{sphinxVerbatim}[commandchars=\\\{\}]
\PYGZdl{}\PYG{+w}{ }\PYG{n+nb}{cd}\PYG{+w}{ }/mnt/c/Users/\PYGZlt{}Username\PYGZgt{}/\PYGZlt{}project\PYGZhy{}install\PYGZhy{}path\PYGZgt{}
\end{sphinxVerbatim}

\sphinxAtStartPar
This pipeline has been tested on WSL2, Ubuntu 22.04. Though any debian\sphinxhyphen{}based distribution should work.


\subsubsection{For unix (Linux/MacOS):}
\label{\detokenize{get_started/install:for-unix-linux-macos}}
\sphinxAtStartPar
In Linux, Mac, WSL or mysys, clone this repository with the pre\sphinxhyphen{}installed git client:

\begin{sphinxVerbatim}[commandchars=\\\{\}]
\PYGZdl{}\PYG{+w}{ }\PYG{n+nb}{cd}\PYG{+w}{ }\PYGZti{}/Documents/MATLAB
\PYGZdl{}\PYG{+w}{ }git\PYG{+w}{ }clone\PYG{+w}{ }https://github.com/ru\PYGZhy{}rbo/caiman\PYGZus{}matlab.git
\PYGZdl{}\PYG{+w}{ }\PYG{n+nb}{cd}\PYG{+w}{ }caiman\PYGZus{}matlab
\PYGZdl{}\PYG{+w}{ }matlab
\end{sphinxVerbatim}


\paragraph{Find MATLAB Install Location:}
\label{\detokenize{get_started/install:find-matlab-install-location}}
\sphinxAtStartPar
The location of the installation is often in \sphinxtitleref{\textasciitilde{}/Documents/MATLAB/}.
If you put the root directory elsewhere, you will need to navigate to that directory within the matlab GUI.

\sphinxAtStartPar
Modern versions of MATLAB (2017+) solve most Linux/Windows filesystem conflicts.

\sphinxAtStartPar
Generally, the main difference in matlab installations on unix vs windows systems is nothing more than the install path:

\begin{sphinxVerbatim}[commandchars=\\\{\}]
\PYG{n}{Windows} \PYG{p}{(}\PYG{l+m+mi}{64}\PYG{o}{\PYGZhy{}}\PYG{n}{bit}\PYG{p}{)}\PYG{p}{:}
\PYG{o}{\PYGZhy{}} \PYG{n}{C}\PYG{p}{:}\PYGZbs{}\PYG{n}{Program} \PYG{n}{Files}\PYGZbs{}\PYG{n}{MATLAB}\PYGZbs{}\PYG{n}{R20XXx} \PYG{p}{(}\PYG{l+m+mi}{64}\PYG{o}{\PYGZhy{}}\PYG{n}{bit} \PYG{n}{MATLAB}\PYG{p}{)}
\PYG{o}{\PYGZhy{}} \PYG{n}{C}\PYG{p}{:}\PYGZbs{}\PYG{n}{Program} \PYG{n}{Files} \PYG{p}{(}\PYG{n}{x86}\PYG{p}{)}\PYGZbs{}\PYG{n}{MATLAB}\PYGZbs{}\PYG{n}{R20XXx} \PYG{p}{(}\PYG{l+m+mi}{32}\PYG{o}{\PYGZhy{}}\PYG{n}{bit} \PYG{n}{MATLAB}\PYG{p}{)}
\PYG{n}{Windows} \PYG{p}{(}\PYG{l+m+mi}{32}\PYG{o}{\PYGZhy{}}\PYG{n}{bit}\PYG{p}{)}\PYG{p}{:}
\PYG{o}{\PYGZhy{}} \PYG{n}{C}\PYG{p}{:}\PYGZbs{}\PYG{n}{Program} \PYG{n}{Files}\PYGZbs{}\PYG{n}{MATLAB}\PYGZbs{}\PYG{n}{R20XXx}
\PYG{n}{Linux}\PYG{p}{:}
\PYG{o}{\PYGZhy{}} \PYG{o}{/}\PYG{n}{usr}\PYG{o}{/}\PYG{n}{local}\PYG{o}{/}\PYG{n}{MATLAB}\PYG{o}{/}\PYG{n}{R20XXx}
\PYG{n}{Mac}\PYG{p}{:}
\PYG{o}{\PYGZhy{}} \PYG{o}{/}\PYG{n}{Applications}\PYG{o}{/}\PYG{n}{MATLAB\PYGZus{}R20XXx}\PYG{o}{.}\PYG{n}{app}
\end{sphinxVerbatim}

\sphinxAtStartPar
To find your install location:

\begin{sphinxVerbatim}[commandchars=\\\{\}]
\PYG{o}{\PYGZgt{}}\PYG{o}{\PYGZgt{}}\PYG{+w}{ }\PYG{n+nb}{matlabroot}
\PYG{+w}{    }\PYG{n+nb}{ans}\PYG{+w}{ }\PYG{p}{=}
\PYG{+w}{        }\PYG{l+s}{\PYGZsq{}}\PYG{l+s}{C:\PYGZbs{}Program Files\PYGZbs{}MATLAB\PYGZbs{}R2023b\PYGZsq{}}
\end{sphinxVerbatim}

\sphinxAtStartPar
Generally, MATLAB code should be stored in your \sphinxtitleref{userpath}:

\begin{sphinxVerbatim}[commandchars=\\\{\}]
\PYG{o}{\PYGZgt{}}\PYG{o}{\PYGZgt{}}\PYG{+w}{ }\PYG{n+nb}{userpath}
\PYG{n+nb}{ans}\PYG{+w}{ }\PYG{p}{=}
\PYG{+w}{    }\PYG{l+s}{\PYGZsq{}}\PYG{l+s}{C:\PYGZbs{}Users\PYGZbs{}RBO\PYGZbs{}Documents\PYGZbs{}MATLAB\PYGZsq{}}
\end{sphinxVerbatim}

\sphinxAtStartPar
Otherwise, you will need to navigate to that directory within the matlab GUI or add the path to this repository:

\begin{sphinxVerbatim}[commandchars=\\\{\}]
\PYG{o}{\PYGZgt{}}\PYG{o}{\PYGZgt{}}\PYG{+w}{ }\PYG{n+nb}{addpath}\PYG{p}{(}\PYG{n+nb}{genpath}\PYG{p}{(}\PYG{l+s}{\PYGZdq{}path/to/caiman\PYGZus{}matlab\PYGZdq{}}\PYG{p}{)}\PYG{p}{)}
\end{sphinxVerbatim}

\sphinxstepscope


\subsection{Semantics}
\label{\detokenize{get_started/semantics:semantics}}\label{\detokenize{get_started/semantics::doc}}
\sphinxAtStartPar
Throughout the pipeline, image\sphinxhyphen{}processing terms will be used such as:

\begin{sphinxVerbatim}[commandchars=\\\{\}]
\PYG{n}{ROI} \PYG{n}{STRIP} \PYG{n}{FRAME} \PYG{n}{IMAGE} \PYG{n}{PLANE} \PYG{n}{VOLUME} \PYG{n}{TIME}\PYG{o}{\PYGZhy{}}\PYG{n}{SERIES} \PYG{n}{Z}\PYG{o}{\PYGZhy{}}\PYG{n}{STACK} \PYG{n}{T}\PYG{o}{\PYGZhy{}}\PYG{n}{STACK}
\end{sphinxVerbatim}

\sphinxAtStartPar
This section aims to define these terms with respect to the LBM Data Processing Pipeline.


\begin{savenotes}\sphinxattablestart
\sphinxthistablewithglobalstyle
\centering
\sphinxcapstartof{table}
\sphinxthecaptionisattop
\sphinxcaption{LBM Semantics}\label{\detokenize{get_started/semantics:id1}}
\sphinxaftertopcaption
\begin{tabulary}{\linewidth}[t]{TT}
\sphinxtoprule
\sphinxstyletheadfamily 
\sphinxAtStartPar
Dimension
&\sphinxstyletheadfamily 
\sphinxAtStartPar
Description
\\
\sphinxmidrule
\sphinxtableatstartofbodyhook
\sphinxAtStartPar
{[}X, Y{]}
&
\sphinxAtStartPar
Image / Plane / Frame (short for “picture frame”)
\\
\sphinxhline
\sphinxAtStartPar
{[}X, Y, Z{]}
&
\sphinxAtStartPar
Volume, 3D\sphinxhyphen{}Stack, Z\sphinxhyphen{}Stack
\\
\sphinxhline
\sphinxAtStartPar
{[}X, Y, Z, T{]}
&
\sphinxAtStartPar
Volumetric Timeseries
\\
\sphinxhline
\sphinxAtStartPar
{[}X, Y, T{]}
&
\sphinxAtStartPar
Time\sphinxhyphen{}Series of a 2D Plane
\\
\sphinxbottomrule
\end{tabulary}
\sphinxtableafterendhook\par
\sphinxattableend\end{savenotes}

\sphinxAtStartPar
\sphinxstyleemphasis{Strip} and ROI are used interchangably in the ScanImage documentation. We will be referring to the individual sections of a ScanImage \sphinxtitleref{.tiff} recording as \sphinxtitleref{strips} and refrain from using \sphinxtitleref{ROI}.

\sphinxstepscope


\subsection{Getting Started}
\label{\detokenize{get_started/getting_started:getting-started}}\label{\detokenize{get_started/getting_started::doc}}

\subsubsection{Dependencies}
\label{\detokenize{get_started/getting_started:dependencies}}
\sphinxAtStartPar
Before running your first dataset, you should ensure that all dependencies of the pipeline are satisfied.

\sphinxAtStartPar
This pipeline requires the parallel pool, statistics and machine learning, and image processing toolboxes.

\sphinxAtStartPar
To see what toolboxes you have installed, use \sphinxcode{\sphinxupquote{ver}} in the MATLAB command window:

\begin{sphinxVerbatim}[commandchars=\\\{\}]
\PYG{o}{\PYGZgt{}}\PYG{o}{\PYGZgt{}}\PYG{+w}{ }\PYG{n+nb}{ver}
\PYG{+w}{ }\PYG{o}{\PYGZhy{}}\PYG{o}{\PYGZhy{}}\PYG{o}{\PYGZhy{}}\PYG{o}{\PYGZhy{}}\PYG{o}{\PYGZhy{}}\PYG{o}{\PYGZhy{}}\PYG{o}{\PYGZhy{}}\PYG{o}{\PYGZhy{}}\PYG{o}{\PYGZhy{}}\PYG{o}{\PYGZhy{}}\PYG{o}{\PYGZhy{}}\PYG{o}{\PYGZhy{}}\PYG{o}{\PYGZhy{}}\PYG{o}{\PYGZhy{}}\PYG{o}{\PYGZhy{}}\PYG{o}{\PYGZhy{}}\PYG{o}{\PYGZhy{}}\PYG{o}{\PYGZhy{}}\PYG{o}{\PYGZhy{}}\PYG{o}{\PYGZhy{}}\PYG{o}{\PYGZhy{}}\PYG{o}{\PYGZhy{}}\PYG{o}{\PYGZhy{}}\PYG{o}{\PYGZhy{}}\PYG{o}{\PYGZhy{}}\PYG{o}{\PYGZhy{}}\PYG{o}{\PYGZhy{}}\PYG{o}{\PYGZhy{}}\PYG{o}{\PYGZhy{}}\PYG{o}{\PYGZhy{}}\PYG{o}{\PYGZhy{}}\PYG{o}{\PYGZhy{}}\PYG{o}{\PYGZhy{}}\PYG{o}{\PYGZhy{}}\PYG{o}{\PYGZhy{}}\PYG{o}{\PYGZhy{}}\PYG{o}{\PYGZhy{}}\PYG{o}{\PYGZhy{}}\PYG{o}{\PYGZhy{}}\PYG{o}{\PYGZhy{}}\PYG{o}{\PYGZhy{}}\PYG{o}{\PYGZhy{}}\PYG{o}{\PYGZhy{}}\PYG{o}{\PYGZhy{}}\PYG{o}{\PYGZhy{}}\PYG{o}{\PYGZhy{}}\PYG{o}{\PYGZhy{}}\PYG{o}{\PYGZhy{}}\PYG{o}{\PYGZhy{}}\PYG{o}{\PYGZhy{}}\PYG{o}{\PYGZhy{}}\PYG{o}{\PYGZhy{}}\PYG{o}{\PYGZhy{}}\PYG{o}{\PYGZhy{}}\PYG{o}{\PYGZhy{}}\PYG{o}{\PYGZhy{}}\PYG{o}{\PYGZhy{}}\PYG{o}{\PYGZhy{}}\PYG{o}{\PYGZhy{}}\PYG{o}{\PYGZhy{}}\PYG{o}{\PYGZhy{}}\PYG{o}{\PYGZhy{}}\PYG{o}{\PYGZhy{}}\PYG{o}{\PYGZhy{}}\PYG{o}{\PYGZhy{}}\PYG{o}{\PYGZhy{}}\PYG{o}{\PYGZhy{}}\PYG{o}{\PYGZhy{}}\PYG{o}{\PYGZhy{}}\PYG{o}{\PYGZhy{}}\PYG{o}{\PYGZhy{}}\PYG{o}{\PYGZhy{}}\PYG{o}{\PYGZhy{}}\PYG{o}{\PYGZhy{}}\PYG{o}{\PYGZhy{}}\PYG{o}{\PYGZhy{}}\PYG{o}{\PYGZhy{}}\PYG{o}{\PYGZhy{}}\PYG{o}{\PYGZhy{}}\PYG{o}{\PYGZhy{}}\PYG{o}{\PYGZhy{}}\PYG{o}{\PYGZhy{}}\PYG{o}{\PYGZhy{}}\PYG{o}{\PYGZhy{}}\PYG{o}{\PYGZhy{}}\PYG{o}{\PYGZhy{}}\PYG{o}{\PYGZhy{}}\PYG{o}{\PYGZhy{}}\PYG{o}{\PYGZhy{}}\PYG{o}{\PYGZhy{}}\PYG{o}{\PYGZhy{}}\PYG{o}{\PYGZhy{}}\PYG{o}{\PYGZhy{}}\PYG{o}{\PYGZhy{}}\PYG{o}{\PYGZhy{}}\PYG{o}{\PYGZhy{}}\PYG{o}{\PYGZhy{}}\PYG{o}{\PYGZhy{}}\PYG{o}{\PYGZhy{}}\PYG{o}{\PYGZhy{}}\PYG{o}{\PYGZhy{}}\PYG{o}{\PYGZhy{}}\PYG{o}{\PYGZhy{}}\PYG{o}{\PYGZhy{}}\PYG{o}{\PYGZhy{}}\PYG{o}{\PYGZhy{}}\PYG{o}{\PYGZhy{}}\PYG{o}{\PYGZhy{}}\PYG{o}{\PYGZhy{}}\PYG{o}{\PYGZhy{}}\PYG{o}{\PYGZhy{}}\PYG{o}{\PYGZhy{}}
\PYG{+w}{ }\PYG{n}{MATLAB}\PYG{+w}{ }\PYG{n}{Version}\PYG{p}{:}\PYG{+w}{ }\PYG{l+m+mf}{24.1}\PYG{p}{.}\PYG{l+m+mf}{0.2537033}\PYG{+w}{ }\PYG{p}{(}\PYG{n}{R2024a}\PYG{p}{)}
\PYG{+w}{ }\PYG{n}{MATLAB}\PYG{+w}{ }\PYG{n}{License}\PYG{+w}{ }\PYG{n}{Number}\PYG{p}{:}\PYG{+w}{ }\PYG{l+m+mi}{41007384}
\PYG{+w}{ }\PYG{n}{Operating}\PYG{+w}{ }\PYG{n}{System}\PYG{p}{:}\PYG{+w}{ }\PYG{n}{Linux}\PYG{+w}{ }\PYG{l+m+mf}{6.2}\PYG{p}{.}\PYG{l+m+mi}{0}\PYG{o}{\PYGZhy{}}\PYG{l+m+mi}{36}\PYG{o}{\PYGZhy{}}\PYG{n}{generic}\PYG{+w}{ }\PYGZsh{}\PYG{l+m+mi}{37}\PYG{o}{\PYGZti{}}\PYG{l+m+mf}{22.04}\PYG{p}{.}\PYG{l+m+mi}{1}\PYG{o}{\PYGZhy{}}\PYG{n}{Ubuntu}\PYG{+w}{ }\PYG{n}{SMP}\PYG{+w}{ }\PYG{n}{PREEMPT\PYGZus{}DYNAMIC}\PYG{+w}{ }\PYG{n}{Mon}\PYG{+w}{ }\PYG{n}{Oct}\PYG{+w}{  }\PYG{l+m+mi}{9}\PYG{+w}{ }\PYG{l+m+mi}{15}\PYG{p}{:}\PYG{l+m+mi}{34}\PYG{p}{:}\PYG{l+m+mi}{04}\PYG{+w}{ }\PYG{n}{UTC}\PYG{+w}{ }\PYG{l+m+mi}{2}\PYG{+w}{ }\PYG{n}{x86\PYGZus{}64}
\PYG{+w}{ }\PYG{n}{Java}\PYG{+w}{ }\PYG{n}{Version}\PYG{p}{:}\PYG{+w}{ }\PYG{n}{Java}\PYG{+w}{ }\PYG{l+m+mf}{1.8}\PYG{p}{.}0\PYG{n}{\PYGZus{}202}\PYG{o}{\PYGZhy{}}\PYG{n}{b08}\PYG{+w}{ }\PYG{n}{with}\PYG{+w}{ }\PYG{n}{Oracle}\PYG{+w}{ }\PYG{n}{Corporation}\PYG{+w}{ }\PYG{n}{Java}\PYG{+w}{ }\PYG{n}{HotSpot}\PYG{p}{(}\PYG{n}{TM}\PYG{p}{)}\PYG{+w}{ }\PYG{l+m+mi}{64}\PYG{o}{\PYGZhy{}}\PYG{n}{Bit}\PYG{+w}{ }\PYG{n}{Server}\PYG{+w}{ }\PYG{n}{VM}\PYG{+w}{ }\PYG{n}{mixed}\PYG{+w}{ }\PYG{n+nb}{mode}
\PYG{+w}{ }\PYG{o}{\PYGZhy{}}\PYG{o}{\PYGZhy{}}\PYG{o}{\PYGZhy{}}\PYG{o}{\PYGZhy{}}\PYG{o}{\PYGZhy{}}\PYG{o}{\PYGZhy{}}\PYG{o}{\PYGZhy{}}\PYG{o}{\PYGZhy{}}\PYG{o}{\PYGZhy{}}\PYG{o}{\PYGZhy{}}\PYG{o}{\PYGZhy{}}\PYG{o}{\PYGZhy{}}\PYG{o}{\PYGZhy{}}\PYG{o}{\PYGZhy{}}\PYG{o}{\PYGZhy{}}\PYG{o}{\PYGZhy{}}\PYG{o}{\PYGZhy{}}\PYG{o}{\PYGZhy{}}\PYG{o}{\PYGZhy{}}\PYG{o}{\PYGZhy{}}\PYG{o}{\PYGZhy{}}\PYG{o}{\PYGZhy{}}\PYG{o}{\PYGZhy{}}\PYG{o}{\PYGZhy{}}\PYG{o}{\PYGZhy{}}\PYG{o}{\PYGZhy{}}\PYG{o}{\PYGZhy{}}\PYG{o}{\PYGZhy{}}\PYG{o}{\PYGZhy{}}\PYG{o}{\PYGZhy{}}\PYG{o}{\PYGZhy{}}\PYG{o}{\PYGZhy{}}\PYG{o}{\PYGZhy{}}\PYG{o}{\PYGZhy{}}\PYG{o}{\PYGZhy{}}\PYG{o}{\PYGZhy{}}\PYG{o}{\PYGZhy{}}\PYG{o}{\PYGZhy{}}\PYG{o}{\PYGZhy{}}\PYG{o}{\PYGZhy{}}\PYG{o}{\PYGZhy{}}\PYG{o}{\PYGZhy{}}\PYG{o}{\PYGZhy{}}\PYG{o}{\PYGZhy{}}\PYG{o}{\PYGZhy{}}\PYG{o}{\PYGZhy{}}\PYG{o}{\PYGZhy{}}\PYG{o}{\PYGZhy{}}\PYG{o}{\PYGZhy{}}\PYG{o}{\PYGZhy{}}\PYG{o}{\PYGZhy{}}\PYG{o}{\PYGZhy{}}\PYG{o}{\PYGZhy{}}\PYG{o}{\PYGZhy{}}\PYG{o}{\PYGZhy{}}\PYG{o}{\PYGZhy{}}\PYG{o}{\PYGZhy{}}\PYG{o}{\PYGZhy{}}\PYG{o}{\PYGZhy{}}\PYG{o}{\PYGZhy{}}\PYG{o}{\PYGZhy{}}\PYG{o}{\PYGZhy{}}\PYG{o}{\PYGZhy{}}\PYG{o}{\PYGZhy{}}\PYG{o}{\PYGZhy{}}\PYG{o}{\PYGZhy{}}\PYG{o}{\PYGZhy{}}\PYG{o}{\PYGZhy{}}\PYG{o}{\PYGZhy{}}\PYG{o}{\PYGZhy{}}\PYG{o}{\PYGZhy{}}\PYG{o}{\PYGZhy{}}\PYG{o}{\PYGZhy{}}\PYG{o}{\PYGZhy{}}\PYG{o}{\PYGZhy{}}\PYG{o}{\PYGZhy{}}\PYG{o}{\PYGZhy{}}\PYG{o}{\PYGZhy{}}\PYG{o}{\PYGZhy{}}\PYG{o}{\PYGZhy{}}\PYG{o}{\PYGZhy{}}\PYG{o}{\PYGZhy{}}\PYG{o}{\PYGZhy{}}\PYG{o}{\PYGZhy{}}\PYG{o}{\PYGZhy{}}\PYG{o}{\PYGZhy{}}\PYG{o}{\PYGZhy{}}\PYG{o}{\PYGZhy{}}\PYG{o}{\PYGZhy{}}\PYG{o}{\PYGZhy{}}\PYG{o}{\PYGZhy{}}\PYG{o}{\PYGZhy{}}\PYG{o}{\PYGZhy{}}\PYG{o}{\PYGZhy{}}\PYG{o}{\PYGZhy{}}\PYG{o}{\PYGZhy{}}\PYG{o}{\PYGZhy{}}\PYG{o}{\PYGZhy{}}\PYG{o}{\PYGZhy{}}\PYG{o}{\PYGZhy{}}\PYG{o}{\PYGZhy{}}\PYG{o}{\PYGZhy{}}\PYG{o}{\PYGZhy{}}\PYG{o}{\PYGZhy{}}\PYG{o}{\PYGZhy{}}\PYG{o}{\PYGZhy{}}\PYG{o}{\PYGZhy{}}\PYG{o}{\PYGZhy{}}\PYG{o}{\PYGZhy{}}\PYG{o}{\PYGZhy{}}\PYG{o}{\PYGZhy{}}\PYG{o}{\PYGZhy{}}
\PYG{+w}{ }\PYG{n}{MATLAB}\PYG{+w}{                                                }\PYG{n}{Version}\PYG{+w}{ }\PYG{l+m+mf}{24.1}\PYG{+w}{        }\PYG{p}{(}\PYG{n}{R2024a}\PYG{p}{)}
\PYG{+w}{ }\PYG{n}{Computer}\PYG{+w}{ }\PYG{n}{Vision}\PYG{+w}{ }\PYG{n}{Toolbox}\PYG{+w}{                               }\PYG{n}{Version}\PYG{+w}{ }\PYG{l+m+mf}{24.1}\PYG{+w}{        }\PYG{p}{(}\PYG{n}{R2024a}\PYG{p}{)}
\PYG{+w}{ }\PYG{n}{Curve}\PYG{+w}{ }\PYG{n}{Fitting}\PYG{+w}{ }\PYG{n}{Toolbox}\PYG{+w}{                                 }\PYG{n}{Version}\PYG{+w}{ }\PYG{l+m+mf}{24.1}\PYG{+w}{        }\PYG{p}{(}\PYG{n}{R2024a}\PYG{p}{)}
\PYG{+w}{ }\PYG{n}{Global}\PYG{+w}{ }\PYG{n}{Optimization}\PYG{+w}{ }\PYG{n}{Toolbox}\PYG{+w}{                           }\PYG{n}{Version}\PYG{+w}{ }\PYG{l+m+mf}{24.1}\PYG{+w}{        }\PYG{p}{(}\PYG{n}{R2024a}\PYG{p}{)}
\PYG{+w}{ }\PYG{n}{Image}\PYG{+w}{ }\PYG{n}{Processing}\PYG{+w}{ }\PYG{n}{Toolbox}\PYG{+w}{                              }\PYG{n}{Version}\PYG{+w}{ }\PYG{l+m+mf}{24.1}\PYG{+w}{        }\PYG{p}{(}\PYG{n}{R2024a}\PYG{p}{)}
\PYG{+w}{ }\PYG{n}{Optimization}\PYG{+w}{ }\PYG{n}{Toolbox}\PYG{+w}{                                  }\PYG{n}{Version}\PYG{+w}{ }\PYG{l+m+mf}{24.1}\PYG{+w}{        }\PYG{p}{(}\PYG{n}{R2024a}\PYG{p}{)}
\PYG{+w}{ }\PYG{n}{Parallel}\PYG{+w}{ }\PYG{n}{Computing}\PYG{+w}{ }\PYG{n}{Toolbox}\PYG{+w}{                            }\PYG{n}{Version}\PYG{+w}{ }\PYG{l+m+mf}{24.1}\PYG{+w}{        }\PYG{p}{(}\PYG{n}{R2024a}\PYG{p}{)}
\PYG{+w}{ }\PYG{n}{Signal}\PYG{+w}{ }\PYG{n}{Processing}\PYG{+w}{ }\PYG{n}{Toolbox}\PYG{+w}{                             }\PYG{n}{Version}\PYG{+w}{ }\PYG{l+m+mf}{24.1}\PYG{+w}{        }\PYG{p}{(}\PYG{n}{R2024a}\PYG{p}{)}
\PYG{+w}{ }\PYG{n}{Statistics}\PYG{+w}{ }\PYG{n}{and}\PYG{+w}{ }\PYG{n}{Machine}\PYG{+w}{ }\PYG{n}{Learning}\PYG{+w}{ }\PYG{n}{Toolbox}\PYG{+w}{               }\PYG{n}{Version}\PYG{+w}{ }\PYG{l+m+mf}{24.1}\PYG{+w}{        }\PYG{p}{(}\PYG{n}{R2024a}\PYG{p}{)}
\PYG{+w}{ }\PYG{n}{Wavelet}\PYG{+w}{ }\PYG{n}{Toolbox}\PYG{+w}{                                       }\PYG{n}{Version}\PYG{+w}{ }\PYG{l+m+mf}{24.1}\PYG{+w}{        }\PYG{p}{(}\PYG{n}{R2024a}\PYG{p}{)}
\end{sphinxVerbatim}

\sphinxAtStartPar
If the user choses to split frames across multiple \sphinxtitleref{.tiff} files, there will be multiple tiff files in ascending order
of an suffix appended to the filename: \sphinxtitleref{\_000N}, where n=number of files chosen by the user.

\begin{sphinxadmonition}{important}{Important:}
\sphinxAtStartPar
All output .tiff files for a single imaging session should be placed in the same directory.
No other .tiff files should be in this directory. If this happens, an error will throw.
\end{sphinxadmonition}


\subsubsection{Directory Structure}
\label{\detokenize{get_started/getting_started:directory-structure}}
\sphinxAtStartPar
The following is an example of the directory hierarchy
used for the demo.

\begin{sphinxVerbatim}[commandchars=\\\{\}]
Parent
├── raw
│   └── basename\PYGZus{}00001\PYGZus{}0001.tiff
│   └── basename\PYGZus{}00001\PYGZus{}0002.tiff
│   └── basename\PYGZus{}00001\PYGZus{}00NN.tiff
├── extraction
│   └── basename.h5
├── registration
│   └── registered\PYGZus{}plane\PYGZus{}1.mat
│   └── registered\PYGZus{}plane\PYGZus{}2.mat
│   └── registered\PYGZus{}plane\PYGZus{}NN.mat
└── segmentation
    └── caiman\PYGZus{}output\PYGZus{}plane\PYGZus{}.h5
\end{sphinxVerbatim}

\sphinxAtStartPar
where \sphinxtitleref{N} = the number of \sphinxtitleref{{[}X, Y, T{]}} planar time\sphinxhyphen{}series.

\sphinxAtStartPar
Following the recommendation described in {\hyperref[\detokenize{get_started/install:install-recommendation}]{\sphinxcrossref{\DUrole{std,std-ref}{Install Recommendation}}}} all necessary functions should already be on your
MATLAB path. If an error is encountered, such as:

\begin{sphinxVerbatim}[commandchars=\\\{\}]
\PYG{n}{Undefined}\PYG{+w}{ }\PYG{l+s}{function}\PYG{+w}{ }\PYG{l+s}{\PYGZsq{}convertScanImageTiffToVolume\PYGZsq{}}\PYG{+w}{ }\PYG{l+s}{for}\PYG{+w}{ }\PYG{l+s}{input}\PYG{+w}{ }\PYG{l+s}{arguments}\PYG{+w}{ }\PYG{l+s}{of}\PYG{+w}{ }\PYG{l+s}{type}\PYG{+w}{ }\PYG{l+s}{\PYGZsq{}char\PYGZsq{}}\PYG{l+s}{.}
\end{sphinxVerbatim}

\sphinxAtStartPar
This means the input is not on your MATLAB path. Add this to the top of the script you are running:
\begin{quote}

\begin{sphinxVerbatim}[commandchars=\\\{\}]
\PYG{p}{[}\PYG{n}{fpath}\PYG{p}{,}\PYG{+w}{ }\PYG{n}{fname}\PYG{p}{,}\PYG{+w}{ }\PYG{o}{\PYGZti{}}\PYG{p}{]}\PYG{+w}{ }\PYG{p}{=}\PYG{+w}{ }\PYG{n+nb}{fileparts}\PYG{p}{(}\PYG{n+nb}{fullfile}\PYG{p}{(}\PYG{n+nb}{mfilename}\PYG{p}{(}\PYG{l+s}{\PYGZsq{}}\PYG{l+s}{fullpath\PYGZsq{}}\PYG{p}{)}\PYG{p}{)}\PYG{p}{)}\PYG{p}{;}\PYG{+w}{ }\PYG{c}{\PYGZpc{} path to this script}
\PYG{n+nb}{addpath}\PYG{p}{(}\PYG{n+nb}{genpath}\PYG{p}{(}\PYG{n+nb}{fullfile}\PYG{p}{(}\PYG{n}{fpath}\PYG{p}{,}\PYG{+w}{ }\PYG{l+s}{\PYGZsq{}}\PYG{l+s}{core/\PYGZsq{}}\PYG{p}{)}\PYG{p}{)}\PYG{p}{)}\PYG{p}{;}
\end{sphinxVerbatim}
\end{quote}

\sphinxAtStartPar
You can make sure all of the requirements for the package are in the path with the following:
\begin{quote}

\begin{sphinxVerbatim}[commandchars=\\\{\}]
\PYG{n}{result}\PYG{+w}{ }\PYG{p}{=}\PYG{+w}{ }\PYG{n}{validateRequirements}\PYG{p}{(}\PYG{p}{)}\PYG{p}{;}\PYG{+w}{ }\PYG{c}{\PYGZpc{} make sure we have dependencies in accessible places}
\PYG{k}{if}\PYG{+w}{ }\PYG{n+nb}{ischar}\PYG{p}{(}\PYG{n}{result}\PYG{p}{)}
\PYG{+w}{    }\PYG{n+nb}{error}\PYG{p}{(}\PYG{n}{result}\PYG{p}{)}\PYG{p}{;}
\PYG{k}{else}
\PYG{+w}{    }\PYG{n+nb}{disp}\PYG{p}{(}\PYG{l+s}{\PYGZsq{}}\PYG{l+s}{Proceeding with execution...\PYGZsq{}}\PYG{p}{)}\PYG{p}{;}
\PYG{k}{end}
\end{sphinxVerbatim}
\end{quote}

\sphinxAtStartPar
It is helpful to first set\sphinxhyphen{}up directories where youd like your results to go. Each core function in this pipeline takes a “data” path and a “save” path as arguments. Following the {\hyperref[\detokenize{get_started/getting_started:directory-structure}]{\sphinxcrossref{\DUrole{std,std-ref}{Directory Structure}}}}:

\begin{sphinxVerbatim}[commandchars=\\\{\}]
\PYG{n}{parentpath}\PYG{+w}{ }\PYG{p}{=}\PYG{+w}{ }\PYG{l+s}{\PYGZsq{}}\PYG{l+s}{C:\PYGZbs{}Users\PYGZbs{}RBO\PYGZbs{}Documents\PYGZbs{}data\PYGZbs{}bi\PYGZus{}hemisphere\PYGZbs{}\PYGZsq{}}\PYG{p}{;}\PYG{+w}{ }\PYG{c}{\PYGZpc{} outer directory}
\PYG{n}{raw\PYGZus{}path}\PYG{+w}{ }\PYG{p}{=}\PYG{+w}{ }\PYG{p}{[}\PYG{+w}{ }\PYG{n}{parentpath}\PYG{+w}{ }\PYG{l+s}{\PYGZsq{}}\PYG{l+s}{raw\PYGZbs{}\PYGZsq{}}\PYG{p}{]}\PYG{p}{;}\PYG{+w}{ }\PYG{c}{\PYGZpc{} raw .tiff files live here}
\PYG{n}{extract\PYGZus{}path}\PYG{+w}{ }\PYG{p}{=}\PYG{+w}{ }\PYG{p}{[}\PYG{+w}{ }\PYG{n}{parentpath}\PYG{+w}{ }\PYG{l+s}{\PYGZsq{}}\PYG{l+s}{extracted\PYGZbs{}\PYGZsq{}}\PYG{p}{]}\PYG{p}{;}\PYG{+w}{ }\PYG{c}{\PYGZpc{} re\PYGZhy{}assembled 4D volumetric time\PYGZhy{}series live here}
\PYG{n+nb}{mkdir}\PYG{p}{(}\PYG{n}{extract\PYGZus{}path}\PYG{p}{)}\PYG{p}{;}\PYG{+w}{ }\PYG{n+nb}{mkdir}\PYG{p}{(}\PYG{n}{raw\PYGZus{}path}\PYG{p}{)}\PYG{p}{;}\PYG{+w}{  }\PYG{c}{\PYGZpc{} create these paths}
\end{sphinxVerbatim}

\sphinxAtStartPar
This produces a nicely organized output structure:

\noindent{\hspace*{\fill}\sphinxincludegraphics{{output_paths1}.png}\hspace*{\fill}}

\sphinxstepscope


\section{User Guide}
\label{\detokenize{user_guide/index:user-guide}}\label{\detokenize{user_guide/index::doc}}
\sphinxAtStartPar
This section covers each step in the pipeline in more detail.

\sphinxstepscope


\subsection{Pre\sphinxhyphen{}Processing}
\label{\detokenize{user_guide/pre_processing:pre-processing}}\label{\detokenize{user_guide/pre_processing::doc}}
\sphinxAtStartPar
Before beginning pre\sphinxhyphen{}processing, follow setup steps in {\hyperref[\detokenize{get_started/getting_started:getting-started}]{\sphinxcrossref{\DUrole{std,std-ref}{Getting Started}}}} to make sure the pipeline and dependencies are installed properly.
See {\hyperref[\detokenize{user_guide/troubleshooting:troubleshooting}]{\sphinxcrossref{\DUrole{std,std-ref}{Troubleshooting}}}} for common issues.

\sphinxAtStartPar
Pre\sphinxhyphen{}processing LBM datasets consists of 2 main processing steps:
\begin{enumerate}
\sphinxsetlistlabels{\arabic}{enumi}{enumii}{}{.}%
\item {} 
\sphinxAtStartPar
Reshaping vertically concatenated strips into horizontally concatenated strips

\item {} 
\sphinxAtStartPar
Piecewise motion\sphinxhyphen{}correction

\end{enumerate}


\subsubsection{1. Re\sphinxhyphen{}Construct Volumetric Time\sphinxhyphen{}Series}
\label{\detokenize{user_guide/pre_processing:re-construct-volumetric-time-series}}\label{\detokenize{user_guide/pre_processing:pipeline-step-1}}
\sphinxAtStartPar
Before processing starts, the raw scanimage output needs to be reconstructed to form a correctly\sphinxhyphen{}ordered time\sphinxhyphen{}series.
This is accomplished through the use of {\hyperref[\detokenize{api/core:convertScanImageTiffToVolume}]{\sphinxcrossref{\sphinxcode{\sphinxupquote{convertScanImageTiffToVolume()}}}}}.

\sphinxAtStartPar
Shown in the image below is a graphical representation of this reconstruction.

\sphinxAtStartPar
In its raw form (see A in the below figure), ScanImage tiff files are multipage tiffs \sphinxhyphen{} like a book.

\sphinxAtStartPar
Each page is one \sphinxstyleemphasis{image}, but it doesn’t look like an image:

\noindent\sphinxincludegraphics{{abc_strip1}.png}

\begin{DUlineblock}{0em}
\item[] A: In the above image, represents vertically concatenated \sphinxstylestrong{strip} of our image.
\item[] B: Strips are cut and horizontally concatenated.
\item[] C: After a scan\sphinxhyphen{}phase correction, lines between strips become unnoticable (ideally)
\end{DUlineblock}

\sphinxAtStartPar
\sphinxstylestrong{If you were to open up a raw ScanImage .tiff file in ImageJ, you would see a very long, thin bar as is shown in A.}
\begin{itemize}
\item {} 
\sphinxAtStartPar
Each Z\sphinxhyphen{}Plane is written before moving onto the next timestep

\item {} 
\sphinxAtStartPar
z\sphinxhyphen{}plane 1 @ timepoint 1, z\sphinxhyphen{}plane 2 @ timepoint 1, z\sphinxhyphen{}plane 3 @ timepoint 1, etc.

\end{itemize}

\sphinxAtStartPar
Thus, another task {\hyperref[\detokenize{api/core:convertScanImageTiffToVolume}]{\sphinxcrossref{\sphinxcode{\sphinxupquote{convertScanImageTiffToVolume()}}}}} accomplishes are reordering this tiff stack to be:
\begin{itemize}
\item {} 
\sphinxAtStartPar
z\sphinxhyphen{}plane 1 @ timepont 1, z\sphinxhyphen{}plane 1 @ timepoint 2, etc ..

\end{itemize}

\sphinxAtStartPar
The output \sphinxtitleref{volumetric time\sphinxhyphen{}series} has dimensions \sphinxtitleref{{[}Y,X,Z,T{]}}.

\sphinxAtStartPar
If the user chooses to split frames across multiple \sphinxtitleref{.tiff} files, there will be multiple tiff files in ascending order
of a suffix appended to the filename: \sphinxtitleref{\_000N}, where n=number of files chosen by the user.

\begin{sphinxadmonition}{important}{Important:}
\sphinxAtStartPar
All output .tiff files for a single imaging session should be placed in the same directory.
No other .tiff files should be in this directory. If this happens, an error will throw.
\end{sphinxadmonition}


\paragraph{Extraction Input}
\label{\detokenize{user_guide/pre_processing:extraction-input}}
\sphinxAtStartPar
First, we set up our directory paths. You can chain the output of one function to the input of another. Note the path names match {\hyperref[\detokenize{get_started/getting_started:directory-structure}]{\sphinxcrossref{\DUrole{std,std-ref}{Directory Structure}}}}:

\begin{sphinxVerbatim}[commandchars=\\\{\}]
\PYG{n}{parent\PYGZus{}path}\PYG{+w}{ }\PYG{p}{=}\PYG{+w}{ }\PYG{l+s}{\PYGZsq{}}\PYG{l+s}{C:\PYGZbs{}Users\PYGZbs{}\PYGZlt{}username\PYGZgt{}\PYGZbs{}Documents\PYGZbs{}data\PYGZbs{}bi\PYGZus{}hemisphere\PYGZbs{}\PYGZsq{}}\PYG{p}{;}\PYG{+w}{ }\PYG{c}{\PYGZpc{}}
\PYG{n}{raw\PYGZus{}path}\PYG{+w}{ }\PYG{p}{=}\PYG{+w}{ }\PYG{p}{[}\PYG{+w}{ }\PYG{n}{parent\PYGZus{}path}\PYG{+w}{ }\PYG{l+s}{\PYGZsq{}}\PYG{l+s}{raw\PYGZbs{}\PYGZsq{}}\PYG{p}{]}\PYG{p}{;}\PYG{+w}{ }\PYG{c}{\PYGZpc{} where our raw .tiffs go}
\PYG{n}{extract\PYGZus{}path}\PYG{+w}{ }\PYG{p}{=}\PYG{+w}{ }\PYG{p}{[}\PYG{+w}{ }\PYG{n}{parent\PYGZus{}path}\PYG{+w}{ }\PYG{l+s}{\PYGZsq{}}\PYG{l+s}{extracted\PYGZbs{}\PYGZsq{}}\PYG{p}{]}\PYG{p}{;}
\PYG{n+nb}{mkdir}\PYG{p}{(}\PYG{n}{extract\PYGZus{}path}\PYG{p}{)}\PYG{p}{;}\PYG{+w}{ }\PYG{n+nb}{mkdir}\PYG{p}{(}\PYG{n}{raw\PYGZus{}path}\PYG{p}{)}\PYG{p}{;}
\end{sphinxVerbatim}

\begin{sphinxadmonition}{note}{Note:}
\sphinxAtStartPar
The term “parameter” throughout this guide refers to the inputs to each function.
For example, running “help convertScanImageTiffToVolume” in the command window will
show to you and describe the parameters of that function.
\end{sphinxadmonition}

\sphinxAtStartPar
This is all you need to start processing your data. Actually, it’s quite more than you need.

\sphinxAtStartPar
\sphinxtitleref{raw\_path} is where your raw \sphinxtitleref{.tiff} files will be stored and is the first parameter of {\hyperref[\detokenize{api/core:convertScanImageTiffToVolume}]{\sphinxcrossref{\sphinxcode{\sphinxupquote{convertScanImageTiffToVolume()}}}}}.
\sphinxtitleref{extract\_path} is where our data will be saved, and is the second parameter.
\sphinxhyphen{} Your raw and extract path can be in any folder you wish without worry of file\sphinxhyphen{}name conflicts.
\sphinxhyphen{} All future pipeline steps will automatically exclude these files as they will not have the characters \sphinxtitleref{\_plane\_} in the filename.

\begin{sphinxadmonition}{note}{Note:}
\sphinxAtStartPar
Don’t put the characters \sphinxtitleref{\_plane\_} together in your raw/extracted filenames!
\end{sphinxadmonition}

\sphinxAtStartPar
\sphinxtitleref{diagnostic\_flag} is the next parameter, setting this to 1, ‘1’, or true will display the detected files that would be processed, and stop. This is helpful for controlling which files are processed.

\sphinxAtStartPar
\sphinxtitleref{overwrite}, similar to diagnostic flag, can be set to 1, ‘1’, or true to enable overwriting any previously extracted data. Otherwise, a warning will show and no data will be saved.

\sphinxAtStartPar
\sphinxtitleref{fix\_scan\_phasee} is a very important parameter: it attempts to maximize the phase\sphinxhyphen{}correlation between each line (row) of each strip, as shown below.

\noindent\sphinxincludegraphics{{corr_nocorr_phase_example1}.png}

\sphinxAtStartPar
This example shows that shifting every \sphinxstyleemphasis{other} row of pixels +2 (to the right) in our 2D reconstructed image will maximize the correlation between adjacent rows.


\paragraph{Extraction Output}
\label{\detokenize{user_guide/pre_processing:extraction-output}}
\sphinxAtStartPar
Our data are now saved as a single h5 file separated by file and by plane. This storage format
makes it easy to motion correct each 3D planar time\sphinxhyphen{}series individually. We will be processing small patches of the total image,
roughly 20um in parallel, so attempting to process multiple time\sphinxhyphen{}series will drastically slow down NormCorre.
After successfully running {\hyperref[\detokenize{api/core:convertScanImageTiffToVolume}]{\sphinxcrossref{\sphinxcode{\sphinxupquote{convertScanImageTiffToVolume()}}}}}, there will be a single \sphinxtitleref{.h5} file containing extracted data.

\sphinxAtStartPar
You can use \sphinxcode{\sphinxupquote{h5info(h5path)}} in the MATLAB command window to reveal some helpful information about our data.

\sphinxAtStartPar
The following is an example structure of the HDF5 file at the outermost level:

\begin{sphinxVerbatim}[commandchars=\\\{\}]
\PYG{n+nb}{h5info}\PYG{p}{(}\PYG{n}{extract\PYGZus{}path}\PYG{p}{,}\PYG{+w}{ }\PYG{l+s}{\PYGZsq{}}\PYG{l+s}{/\PYGZsq{}}\PYG{p}{)}

\PYG{n}{Filename}\PYG{p}{:}\PYG{+w}{ }\PYG{l+s}{\PYGZsq{}}\PYG{l+s}{C:\PYGZbs{}Users\PYGZbs{}\PYGZlt{}username\PYGZgt{}\PYGZbs{}MH184\PYGZus{}both\PYGZus{}6mm\PYGZus{}FOV\PYGZus{}150\PYGZus{}600um\PYGZus{}depth\PYGZus{}410mW\PYGZus{}9min\PYGZus{}no\PYGZus{}stimuli\PYGZus{}00001\PYGZus{}00001.h5\PYGZsq{}}
\PYG{n}{Name}\PYG{p}{:}\PYG{+w}{ }\PYG{l+s}{\PYGZsq{}}\PYG{l+s}{/\PYGZsq{}}
\PYG{n}{Groups}\PYG{p}{:}
\PYG{+w}{    }\PYG{o}{/}\PYG{n}{file\PYGZus{}1}
\PYG{+w}{    }\PYG{o}{/}\PYG{n}{file\PYGZus{}2}
\PYG{+w}{    }\PYG{o}{/}\PYG{n}{file\PYGZus{}3}
\PYG{n}{Datasets}\PYG{p}{:}\PYG{+w}{ }\PYG{p}{[}\PYG{p}{]}
\PYG{n}{Datatypes}\PYG{p}{:}\PYG{+w}{ }\PYG{p}{[}\PYG{p}{]}
\PYG{n}{Links}\PYG{p}{:}\PYG{+w}{ }\PYG{p}{[}\PYG{p}{]}
\PYG{n}{Attributes}\PYG{p}{:}\PYG{+w}{ }\PYG{p}{[}\PYG{p}{]}
\end{sphinxVerbatim}

\sphinxAtStartPar
We see here that our “parent” group has 3 subgroups corresponding to the number of raw .tiff files. Lets explore one of these “file” subgroups:

\begin{sphinxVerbatim}[commandchars=\\\{\}]
\PYG{o}{\PYGZgt{}}\PYG{o}{\PYGZgt{}}\PYG{+w}{ }\PYG{n+nb}{h5info}\PYG{p}{(}\PYG{n}{extract\PYGZus{}path}\PYG{p}{,}\PYG{+w}{ }\PYG{l+s}{\PYGZsq{}}\PYG{l+s}{/file\PYGZus{}1\PYGZsq{}}\PYG{p}{)}

\PYG{n}{info}\PYG{+w}{ }\PYG{p}{=}

\PYG{+w}{  }\PYG{n+nb}{struct}\PYG{+w}{ }\PYG{n}{with}\PYG{+w}{ }\PYG{n}{fields}\PYG{p}{:}

\PYG{+w}{      }\PYG{n}{Filename}\PYG{p}{:}\PYG{+w}{ }\PYG{l+s}{\PYGZsq{}}\PYG{l+s}{C:\PYGZbs{}Users\PYGZbs{}RBO\PYGZbs{}Documents\PYGZbs{}data\PYGZbs{}bi\PYGZus{}hemisphere\PYGZbs{}extracted\PYGZbs{}MH184\PYGZus{}both\PYGZus{}6mm\PYGZus{}FOV\PYGZus{}150\PYGZus{}600um\PYGZus{}depth\PYGZus{}410mW\PYGZus{}9min\PYGZus{}no\PYGZus{}stimuli\PYGZus{}00001\PYGZus{}00001.h5\PYGZsq{}}
\PYG{+w}{          }\PYG{n}{Name}\PYG{p}{:}\PYG{+w}{ }\PYG{l+s}{\PYGZsq{}}\PYG{l+s}{/file\PYGZus{}1\PYGZsq{}}
\PYG{+w}{        }\PYG{n}{Groups}\PYG{p}{:}\PYG{+w}{ }\PYG{p}{[}\PYG{p}{]}
\PYG{+w}{      }\PYG{n}{Datasets}\PYG{p}{:}\PYG{+w}{ }\PYG{p}{[}\PYG{l+m+mi}{30}×\PYG{l+m+mi}{1}\PYG{+w}{ }\PYG{n+nb}{struct}\PYG{p}{]}
\PYG{+w}{     }\PYG{n}{Datatypes}\PYG{p}{:}\PYG{+w}{ }\PYG{p}{[}\PYG{p}{]}
\PYG{+w}{         }\PYG{n}{Links}\PYG{p}{:}\PYG{+w}{ }\PYG{p}{[}\PYG{p}{]}
\PYG{+w}{    }\PYG{n}{Attributes}\PYG{p}{:}\PYG{+w}{ }\PYG{p}{[}\PYG{p}{]}
\end{sphinxVerbatim}

\sphinxAtStartPar
We see that there are 30 datasets corresponding to each of our Z\sphinxhyphen{}planes, but no groups or attributes. That information is stored within each plane:

\begin{sphinxVerbatim}[commandchars=\\\{\}]
\PYG{n+nb}{h5info}\PYG{p}{(}\PYG{n}{extract\PYGZus{}path}\PYG{p}{,}\PYG{+w}{ }\PYG{l+s}{\PYGZsq{}}\PYG{l+s}{file\PYGZus{}1/plane\PYGZus{}1\PYGZsq{}}\PYG{p}{)}

\PYG{+w}{  }\PYG{n+nb}{struct}\PYG{+w}{ }\PYG{n}{with}\PYG{+w}{ }\PYG{n}{fields}\PYG{p}{:}

\PYG{+w}{  }\PYG{n}{Filename}\PYG{p}{:}\PYG{+w}{ }\PYG{l+s}{\PYGZsq{}}\PYG{l+s}{C:\PYGZbs{}Users\PYGZbs{}\PYGZlt{}username\PYGZgt{}\PYGZbs{}extracted\PYGZbs{}MH184\PYGZus{}both\PYGZus{}6mm\PYGZus{}FOV\PYGZus{}150\PYGZus{}600um\PYGZus{}depth\PYGZus{}410mW\PYGZus{}9min\PYGZus{}no\PYGZus{}stimuli\PYGZus{}00001\PYGZus{}00001.h5\PYGZsq{}}
\PYG{+w}{      }\PYG{n}{Name}\PYG{p}{:}\PYG{+w}{ }\PYG{l+s}{\PYGZsq{}}\PYG{l+s}{plane\PYGZus{}1\PYGZsq{}}
\PYG{+w}{  }\PYG{n}{Datatype}\PYG{p}{:}\PYG{+w}{ }\PYG{p}{[}\PYG{l+m+mi}{1}×\PYG{l+m+mi}{1}\PYG{+w}{ }\PYG{n+nb}{struct}\PYG{p}{]}
\PYG{+w}{ }\PYG{n}{Dataspace}\PYG{p}{:}\PYG{+w}{ }\PYG{p}{[}\PYG{l+m+mi}{1}×\PYG{l+m+mi}{1}\PYG{+w}{ }\PYG{n+nb}{struct}\PYG{p}{]}
\PYG{+w}{ }\PYG{n}{ChunkSize}\PYG{p}{:}\PYG{+w}{ }\PYG{p}{[}\PYG{l+m+mi}{1165}\PYG{+w}{ }\PYG{l+m+mi}{1202}\PYG{+w}{ }\PYG{l+m+mi}{1}\PYG{p}{]}
\PYG{+w}{ }\PYG{n}{FillValue}\PYG{p}{:}\PYG{+w}{ }\PYG{l+m+mi}{0}
\PYG{+w}{   }\PYG{n}{Filters}\PYG{p}{:}\PYG{+w}{ }\PYG{p}{[}\PYG{l+m+mi}{1}×\PYG{l+m+mi}{1}\PYG{+w}{ }\PYG{n+nb}{struct}\PYG{p}{]}
\PYG{n}{Attributes}\PYG{p}{:}\PYG{+w}{ }\PYG{p}{[}\PYG{l+m+mi}{30}×\PYG{l+m+mi}{1}\PYG{+w}{ }\PYG{n+nb}{struct}\PYG{p}{]}
\end{sphinxVerbatim}
\begin{itemize}
\item {} 
\sphinxAtStartPar
\sphinxstylestrong{Groups}: h5 files can be thought of like directories where a 3D time\sphinxhyphen{}series is self contained within its own folder (or group).

\item {} 
\sphinxAtStartPar
\sphinxstylestrong{Attributes}: Attributes are special “tags” attached to a group. This is where we store metadata associated with each group and dataset. The result of calling \sphinxtitleref{get\_metadata(raw\_path)} (see {\hyperref[\detokenize{user_guide/metadata:scanimage-metadata}]{\sphinxcrossref{\DUrole{std,std-ref}{ScanImage Metadata}}}} for more information about the magic behind the scenes here).

\end{itemize}

\sphinxAtStartPar
Due to this organization, to retrieve a 3D time\sphinxhyphen{}series for a single Z\sphinxhyphen{}plane, you must collect individual time\sphinxhyphen{}series from each file.

\sphinxAtStartPar
{\hyperref[\detokenize{api/utils:combinePlanes}]{\sphinxcrossref{\sphinxcode{\sphinxupquote{combinePlanes()}}}}} will do this for you given the path to the h5file and the index of which plane you wish to aquire.

\begin{sphinxVerbatim}[commandchars=\\\{\}]
\PYG{c}{\PYGZpc{} retrieve a 3D time\PYGZhy{}series for the third z\PYGZhy{}plane}
\PYG{n}{z\PYGZus{}time\PYGZus{}series}\PYG{+w}{ }\PYG{p}{=}\PYG{+w}{ }\PYG{n}{combinePlanes}\PYG{p}{(}\PYG{n}{h5path}\PYG{p}{,}\PYG{+w}{ }\PYG{l+m+mi}{3}\PYG{p}{)}\PYG{p}{;}

\PYG{c}{\PYGZpc{} visualize the second timestep}
\PYG{n+nb}{figure}\PYG{p}{;}\PYG{+w}{ }\PYG{n+nb}{imagesc}\PYG{p}{(}\PYG{n}{z\PYGZus{}time\PYGZus{}series}\PYG{p}{(}\PYG{p}{:}\PYG{p}{,}\PYG{p}{:}\PYG{p}{,}\PYG{l+m+mi}{2}\PYG{p}{)}\PYG{p}{)}\PYG{p}{;}\PYG{+w}{ }\PYG{n}{axis}\PYG{+w}{ }\PYG{l+s}{image}\PYG{p}{;}
\end{sphinxVerbatim}

\noindent{\hspace*{\fill}\sphinxincludegraphics{{quickview_blue1}.png}\hspace*{\fill}}


\subsubsection{2. Piecewise\sphinxhyphen{}Rigid Motion\sphinxhyphen{}Correction}
\label{\detokenize{user_guide/pre_processing:piecewise-rigid-motion-correction}}
\sphinxAtStartPar
For a quick demo on how to run motion correction, see the demo\_registration.m script.

\begin{sphinxadmonition}{note}{Note:}
\sphinxAtStartPar
The terms motion\sphinxhyphen{}correction and registration are often used interchangably.
\end{sphinxadmonition}

\sphinxAtStartPar
The goal of motion correction is to make sure that our neuron in the first frame is in the same spatial location as in frame N throughout the time\sphinxhyphen{}series.
Natural movement by the animal during experimental tasks can cause our images spatial potition varying slightly frame by frame. The extent of this movement can also vary widely depending
on the type of task the animal is performing.

\sphinxAtStartPar
For this reason, it is \sphinxstyleemphasis{very} important for the researcher to verify that any motion artifacts in the movie are removed before moving onto any subsequent computations.


\paragraph{Rigid vs Non\sphinxhyphen{}Rigid}
\label{\detokenize{user_guide/pre_processing:rigid-vs-non-rigid}}
\sphinxAtStartPar
The motion artifacts present in a movie also come in two flavors, \sphinxtitleref{rigid} and \sphinxtitleref{non\sphinxhyphen{}rigid}.
Purely rigid motion is simple, straightforeward movement that applies to each\sphinxhyphen{}and\sphinxhyphen{}every pixel equally.
The entire 2D image is shifted by a number of pixels in the x direction and y direction.

\sphinxAtStartPar
Non\sphinxhyphen{}rigid artifacts are much more complex as one region of the 2D image requires shifts that another region does not.

\sphinxAtStartPar
Motion correction relies on \phantomsection\label{\detokenize{user_guide/pre_processing:normcorre}}NoRMCorre for piecewise\sphinxhyphen{}rigid motion correction resulting in shifts for each patch.

\noindent\sphinxincludegraphics{{patches1}.png}

\sphinxAtStartPar
To run motion\sphinxhyphen{}correction, call \sphinxtitleref{motionCorrectPlane()}:

\begin{sphinxVerbatim}[commandchars=\\\{\}]
\PYG{c}{\PYGZpc{} recall our directory structure, chaining the output path from the}
\PYG{c}{\PYGZpc{} tiff reconstruction step}
\PYG{n}{mcpath}\PYG{+w}{ }\PYG{p}{=}\PYG{+w}{ }\PYG{l+s}{\PYGZsq{}}\PYG{l+s}{C:\PYGZbs{}Users\PYGZbs{}RBO\PYGZbs{}Documents\PYGZbs{}data\PYGZbs{}bi\PYGZus{}hemisphere\PYGZbs{}registration\PYGZsq{}}\PYG{p}{;}

\PYG{c}{\PYGZpc{} data\PYGZus{}path, save\PYGZus{}path, num\PYGZus{}cores, start\PYGZus{}plane, end\PYGZus{}plane}
\PYG{n}{motionCorrectPlane}\PYG{p}{(}\PYG{n}{extract\PYGZus{}path}\PYG{p}{,}\PYG{+w}{ }\PYG{n}{mcpath}\PYG{p}{,}\PYG{+w}{ }\PYG{l+m+mi}{23}\PYG{p}{,}\PYG{+w}{ }\PYG{l+m+mi}{1}\PYG{p}{,}\PYG{+w}{ }\PYG{l+m+mi}{3}\PYG{p}{)}\PYG{p}{;}
\end{sphinxVerbatim}

\sphinxAtStartPar
For input, use the same directory as \sphinxtitleref{save\_path} parameter in {\hyperref[\detokenize{api/core:convertScanImageTiffToVolume}]{\sphinxcrossref{\sphinxcode{\sphinxupquote{convertScanImageTiffToVolume()}}}}}.
\begin{itemize}
\item {} 
\sphinxAtStartPar
\sphinxtitleref{data\_path}: Path to your extracted dataset.

\item {} 
\sphinxAtStartPar
\sphinxtitleref{save\_path}: Path to save your data.

\item {} 
\sphinxAtStartPar
\sphinxtitleref{num\_cores}: the number of CPU cores to dedicate to motion\sphinxhyphen{}correction.

\item {} 
\sphinxAtStartPar
\sphinxtitleref{start\_plane}: The index of the first z\sphinxhyphen{}plane to motion\sphinxhyphen{}correct.

\item {} 
\sphinxAtStartPar
\sphinxtitleref{end\_plane}: The index of the last z\sphinxhyphen{}plane to motion\sphinxhyphen{}correct.

\end{itemize}

\begin{sphinxadmonition}{note}{Note:}
\sphinxAtStartPar
Each Z\sphinxhyphen{}plane in between start\_plane and end\_plane will be processed. In the future we may want to provide a way to give an array of indices to correct e.g. if the user wants to throw out Z\sphinxhyphen{}planes.
\end{sphinxadmonition}


\paragraph{Registration Output}
\label{\detokenize{user_guide/pre_processing:registration-output}}\begin{itemize}
\item {} 
\sphinxAtStartPar
The output is a 2D column vector {[}x, y{]} with shifts that allow you to reconstruct the motion\sphinxhyphen{}corrected movie with \phantomsection\label{\detokenize{user_guide/pre_processing:core-utils-translateframes}}core.utils.translateFrames.

\item {} 
\sphinxAtStartPar
shifts(:,1) represent pixel\sphinxhyphen{}shifts in \sphinxstyleemphasis{x}

\item {} 
\sphinxAtStartPar
shifts(:,2) represent pixel\sphinxhyphen{}shifts in \sphinxstyleemphasis{y}

\end{itemize}

\sphinxAtStartPar
Perform both piecewise\sphinxhyphen{}rigid motion correction using {\hyperref[\detokenize{user_guide/pre_processing:normcorre}]{\sphinxcrossref{NormCORRe}}} to stabilize the imaging data. Each plane is motion corrected sequentially, so
only a single plane is ever loaded into memory due to large LBM filesizes (\textgreater{}35GB). A template of 150\sphinxhyphen{}200 frames is used to initialize a “reference image”.

\noindent\sphinxincludegraphics{{template11}.png}

\sphinxAtStartPar
This image is your “ground truth” per\sphinxhyphen{}se, it is the image you want to most accurately represent the movement in your video.

\begin{sphinxVerbatim}[commandchars=\\\{\}]
\PYG{o}{\PYGZgt{}}\PYG{o}{\PYGZgt{}}\PYG{+w}{ }\PYG{n+nb}{help}\PYG{+w}{ }\PYG{n}{translateFrames}

\PYG{+w}{  }\PYG{n}{translateFrames}\PYG{+w}{ }\PYG{n}{Translate}\PYG{+w}{ }\PYG{n+nb}{image}\PYG{+w}{ }\PYG{n}{frames}\PYG{+w}{ }\PYG{n}{based}\PYG{+w}{ }\PYG{n}{on}\PYG{+w}{ }\PYG{n}{provided}\PYG{+w}{ }\PYG{n}{translation}\PYG{+w}{ }\PYG{n}{vectors}\PYG{p}{.}

\PYG{+w}{   }\PYG{n}{This}\PYG{+w}{ }\PYG{k}{function}\PYG{+w}{ }\PYG{n}{applies}\PYG{+w}{ }2\PYG{n}{D}\PYG{+w}{ }\PYG{n}{translations}\PYG{+w}{ }\PYG{n}{to}\PYG{+w}{ }\PYG{n}{an}\PYG{+w}{ }\PYG{n+nb}{image}\PYG{+w}{ }\PYG{n+nb}{time}\PYG{+w}{ }\PYG{n}{series}\PYG{+w}{ }\PYG{n}{based}\PYG{+w}{ }\PYG{n}{on}
\PYG{+w}{   }\PYG{n}{a}\PYG{+w}{ }\PYG{n}{series}\PYG{+w}{ }\PYG{n}{of}\PYG{+w}{ }\PYG{n}{translation}\PYG{+w}{ }\PYG{n}{vectors}\PYG{p}{,}\PYG{+w}{ }\PYG{n}{one}\PYG{+w}{ }\PYG{n}{per}\PYG{+w}{ }\PYG{n}{frame}\PYG{p}{.}\PYG{+w}{ }\PYG{n}{Each}\PYG{+w}{ }\PYG{n}{frame}\PYG{+w}{ }\PYG{n}{is}\PYG{+w}{ }\PYG{n}{translated}
\PYG{+w}{   }\PYG{n}{independently}\PYG{p}{,}\PYG{+w}{ }\PYG{n}{and}\PYG{+w}{ }\PYG{n}{the}\PYG{+w}{ }\PYG{n}{result}\PYG{+w}{ }\PYG{n}{is}\PYG{+w}{ }\PYG{n}{returned}\PYG{+w}{ }\PYG{n}{as}\PYG{+w}{ }\PYG{n}{a}\PYG{+w}{ }3\PYG{n}{D}\PYG{+w}{ }\PYG{n+nb}{stack}\PYG{+w}{ }\PYG{n}{of}
\PYG{+w}{   }\PYG{p}{(}\PYG{n}{Height}\PYG{+w}{ }\PYG{n}{x}\PYG{+w}{ }\PYG{n}{Width}\PYG{+w}{ }\PYG{n}{x}\PYG{+w}{ }\PYG{n}{num\PYGZus{}frames}\PYG{p}{)}\PYG{+w}{ }\PYG{n}{translated}\PYG{+w}{ }\PYG{n}{frames}\PYG{p}{.}

\PYG{+w}{   }\PYG{n}{Inputs}\PYG{p}{:}
\PYG{+w}{     }\PYG{n}{Y}\PYG{+w}{ }\PYG{o}{\PYGZhy{}}\PYG{+w}{ }\PYG{n}{A}\PYG{+w}{ }3\PYG{n}{D}\PYG{+w}{ }\PYG{n+nb}{time}\PYG{+w}{ }\PYG{n}{series}\PYG{+w}{ }\PYG{n}{of}\PYG{+w}{ }\PYG{n+nb}{image}\PYG{+w}{ }\PYG{n}{frames}\PYG{+w}{ }\PYG{p}{(}\PYG{n}{Height}\PYG{+w}{ }\PYG{n}{x}\PYG{+w}{ }\PYG{n}{Width}\PYG{+w}{ }\PYG{n}{x}\PYG{+w}{ }\PYG{n}{Number}\PYG{+w}{ }\PYG{n}{of}\PYG{+w}{ }\PYG{n}{Frames}\PYG{p}{)}\PYG{p}{.}
\PYG{+w}{     }\PYG{n}{t\PYGZus{}shifts}\PYG{+w}{ }\PYG{o}{\PYGZhy{}}\PYG{+w}{ }\PYG{n}{An}\PYG{+w}{ }\PYG{n}{Nx2}\PYG{+w}{ }\PYG{n}{matrix}\PYG{+w}{ }\PYG{n}{of}\PYG{+w}{ }\PYG{n}{translation}\PYG{+w}{ }\PYG{n}{vectors}\PYG{+w}{ }\PYG{k}{for}\PYG{+w}{ }\PYG{n}{each}\PYG{+w}{ }\PYG{n}{frame}\PYG{+w}{ }\PYG{p}{(}\PYG{n}{N}\PYG{+w}{ }\PYG{n}{is}\PYG{+w}{ }\PYG{n}{the}\PYG{+w}{ }\PYG{n}{number}\PYG{+w}{ }\PYG{n}{of}\PYG{+w}{ }\PYG{n}{frames}\PYG{p}{)}\PYG{p}{.}

\PYG{+w}{   }\PYG{n}{Output}\PYG{p}{:}
\PYG{+w}{     }\PYG{n}{translatedFrames}\PYG{+w}{ }\PYG{o}{\PYGZhy{}}\PYG{+w}{ }\PYG{n}{A}\PYG{+w}{ }3\PYG{n}{D}\PYG{+w}{ }\PYG{n}{array}\PYG{+w}{ }\PYG{n}{of}\PYG{+w}{ }\PYG{n}{translated}\PYG{+w}{ }\PYG{n+nb}{image}\PYG{+w}{ }\PYG{n}{frames}\PYG{p}{,}\PYG{+w}{ }\PYG{n}{same}\PYG{+w}{ }\PYG{n+nb}{size}\PYG{+w}{ }\PYG{n}{and}\PYG{+w}{ }\PYG{n+nb}{type}\PYG{+w}{ }\PYG{n}{as}\PYG{+w}{ }\PYG{n}{Y}\PYG{p}{.}
\end{sphinxVerbatim}

\noindent\sphinxincludegraphics{{storage_rec1}.png}


\paragraph{Metrics}
\label{\detokenize{user_guide/pre_processing:metrics}}
\sphinxAtStartPar
CaImAn provides some useful metrics to determine the effectiveness of registration.
These will be placed in the same directory as your save\_path, \sphinxtitleref{metrics/registration\_metrics\_plane\_N}.

\noindent\sphinxincludegraphics{{motion_metrics1}.png}

\sphinxAtStartPar
The top figure shows our shifts for rigid and non\sphinxhyphen{}rigid motion correction. This gives an idea what proportion of the movement corrected for can be attributed to rigid or non\sphinxhyphen{}rigid motion.
Underneath you see the rigid shifts for X and Y, respectively.

\sphinxAtStartPar
To view the video, use the function {\hyperref[\detokenize{api/utils:planeToMovie}]{\sphinxcrossref{\sphinxcode{\sphinxupquote{planeToMovie()}}}}}, which can also zoom in on select areas of your movie:

\begin{sphinxVerbatim}[commandchars=\\\{\}]
\PYG{n}{Inputs}\PYG{p}{:}
  \PYG{n}{data} \PYG{o}{\PYGZhy{}} \PYG{l+m+mi}{3}\PYG{n}{D} \PYG{n}{matrix} \PYG{n}{of} \PYG{n}{image} \PYG{n}{data}\PYG{o}{.}
  \PYG{n}{filename} \PYG{o}{\PYGZhy{}} \PYG{n}{Name} \PYG{n}{of} \PYG{n}{the} \PYG{n}{output} \PYG{n}{video} \PYG{n}{file}\PYG{o}{.}
  \PYG{n}{x} \PYG{o}{\PYGZhy{}} \PYG{n}{Horizontal} \PYG{n}{coordinates}\PYG{o}{.}
  \PYG{n}{y} \PYG{o}{\PYGZhy{}} \PYG{n}{Vertical} \PYG{n}{coordinates}\PYG{o}{.}
  \PYG{n}{frameRate} \PYG{o}{\PYGZhy{}} \PYG{n}{Frame} \PYG{n}{rate} \PYG{n}{of} \PYG{n}{the} \PYG{n}{output} \PYG{n}{video}\PYG{o}{.}
  \PYG{n}{avgs} \PYG{o}{\PYGZhy{}} \PYG{n}{Number} \PYG{n}{of} \PYG{n}{frames} \PYG{n}{to} \PYG{n}{average}\PYG{o}{.}
  \PYG{n}{zoom} \PYG{o}{\PYGZhy{}} \PYG{n}{Zoom} \PYG{n}{factors}\PYG{o}{.} \PYG{p}{(}\PYG{o+ow}{not} \PYG{n}{implemented}\PYG{p}{)}
  \PYG{n}{decenter} \PYG{o}{\PYGZhy{}} \PYG{n}{Decentering} \PYG{n}{offsets}\PYG{o}{.}
  \PYG{n}{crf} \PYG{o}{\PYGZhy{}} \PYG{n}{Constant} \PYG{n}{Rate} \PYG{n}{Factor} \PYG{k}{for} \PYG{n}{video} \PYG{n}{quality}\PYG{o}{.}
\end{sphinxVerbatim}

\sphinxstepscope


\subsection{Segmentation and Deconvolution}
\label{\detokenize{user_guide/segmentation:segmentation-and-deconvolution}}\label{\detokenize{user_guide/segmentation:segmentation-deconvolution}}\label{\detokenize{user_guide/segmentation::doc}}
\noindent\sphinxincludegraphics{{neuron_to_neuron_correlations1}.png}


\subsubsection{Overview}
\label{\detokenize{user_guide/segmentation:overview}}
\sphinxAtStartPar
The flourescence of the proteins in our neurons is \sphinxstylestrong{correlated} with how active the neuron is.
Turning this flourescence into “spikes” relies on several operations:
\begin{itemize}
\item {} 
\sphinxAtStartPar
Use \sphinxtitleref{matfile} to load parameters of the motion\sphinxhyphen{}corrected movie without loading the entire movie into memory.

\item {} 
\sphinxAtStartPar
Set parameters for CNMF.

\item {} 
\sphinxAtStartPar
Perform patched, piecewise, volumetric CNMF.

\item {} 
\sphinxAtStartPar
Save outputs containing both neuropil and accepted/rejected components.

\end{itemize}

\sphinxAtStartPar
segmentPlane.m
\sphinxhyphen{} construct\_patches.m (caiman)
\sphinxhyphen{} run\_CNMF\_patches.m (caiman)
\sphinxhyphen{} classify\_components.m (caiman)
\sphinxhyphen{} compute\_event\_exceptionality (caiman)
\sphinxhyphen{} update\_temporal\_components (caiman)
\sphinxhyphen{} detrend\_df\_f (caiman \sphinxhyphen{} modified)
\sphinxhyphen{} AtoAc

\sphinxAtStartPar
Dependencies: CaImAn\_Utilities

\begin{sphinxadmonition}{note}{Note:}
\sphinxAtStartPar
No time threshold is used for component validation.
\end{sphinxadmonition}


\subsubsection{Function Usage}
\label{\detokenize{user_guide/segmentation:function-usage}}

\paragraph{segmentPlane}
\label{\detokenize{user_guide/segmentation:segmentplane}}
\sphinxAtStartPar
{\hyperref[\detokenize{user_guide/segmentation:segmentplane}]{\sphinxcrossref{\DUrole{std,std-ref}{segmentPlane}}}} contains the bulk of the computational complexity in this pipeline and will take significantly longer than the previous steps.

\sphinxAtStartPar
Inputs to {\hyperref[\detokenize{user_guide/segmentation:segmentplane}]{\sphinxcrossref{\DUrole{std,std-ref}{segmentPlane}}}} are similar to those seen previously. Most importantly:
\begin{itemize}
\item {} 
\sphinxAtStartPar
data\_path: full path to motion\sphinxhyphen{}corrected \sphinxtitleref{.mat} files:

\begin{sphinxVerbatim}[commandchars=\\\{\}]
\PYG{n}{data\PYGZus{}path} \PYG{n}{should} \PYG{n}{contain} \PYG{n}{a} \PYG{n}{single} \PYG{o}{.}\PYG{n}{mat} \PYG{n}{file} \PYG{k}{for} \PYG{o}{*}\PYG{o}{*}\PYG{n}{each} \PYG{l+m+mi}{3}\PYG{n}{D} \PYG{n}{planar} \PYG{n}{time}\PYG{o}{\PYGZhy{}}\PYG{n}{series}\PYG{o}{*}\PYG{o}{*}

\PYG{n}{The} \PYG{n}{pipeline} \PYG{n}{forces} \PYG{n}{filenames} \PYG{n}{to} \PYG{n}{contain} \PYG{n}{\PYGZus{}plane\PYGZus{}N}\PYG{o}{.}\PYG{n}{mat}\PYG{p}{,} \PYG{n}{which} \PYG{o+ow}{is} \PYG{n}{used} \PYG{n}{to} \PYG{n}{isolate} \PYG{n}{only} \PYG{n}{motion}\PYG{o}{\PYGZhy{}}\PYG{n}{corrected}
\PYG{n}{mat} \PYG{n}{files} \PYG{k}{for} \PYG{n}{this} \PYG{n}{session}\PYG{o}{.}
\end{sphinxVerbatim}

\item {} 
\sphinxAtStartPar
save\_path: full path to a folder where you would like to save the neuron/neuropil components and traces:

\begin{sphinxVerbatim}[commandchars=\\\{\}]
\PYG{n}{This} \PYG{n}{will}\PYG{p}{,} \PYG{k}{as} \PYG{n}{of} \PYG{n}{v0}\PYG{l+m+mf}{.2}\PYG{l+m+mf}{.0}\PYG{p}{,} \PYG{n}{overwrite} \PYG{n+nb}{any} \PYG{n}{previously} \PYG{n}{obtained} \PYG{n}{data} \PYG{o+ow}{in} \PYG{n}{this} \PYG{n}{folder}\PYG{o}{.}
\end{sphinxVerbatim}

\end{itemize}

\sphinxAtStartPar
When running {\hyperref[\detokenize{user_guide/segmentation:segmentplane}]{\sphinxcrossref{\DUrole{std,std-ref}{segmentPlane}}}}, check the command window for reports that match the number of files you expect to be processed:

\begin{sphinxVerbatim}[commandchars=\\\{\}]
\PYG{n}{Processing}\PYG{+w}{ }\PYG{l+s}{30}\PYG{+w}{ }\PYG{l+s}{files}\PYG{+w}{ }\PYG{l+s}{found}\PYG{+w}{ }\PYG{l+s}{in}\PYG{+w}{ }\PYG{l+s}{directory}\PYG{+w}{ }\PYG{l+s}{C:\PYGZbs{}Users\PYGZbs{}\PYGZlt{}username\PYGZgt{}\PYGZbs{}Documents\PYGZbs{}data\PYGZbs{}bi\PYGZus{}hemisphere\PYGZbs{}registration\PYGZbs{}...}\PYG{+w}{  }\PYG{l+s}{\PYGZpc{}\PYGZpc{}}\PYG{+w}{ }\PYG{l+s}{our}\PYG{+w}{ }\PYG{l+s}{data\PYGZus{}path}
\PYG{n}{Beginning}\PYG{+w}{ }\PYG{l+s}{calculations}\PYG{+w}{ }\PYG{l+s}{for}\PYG{+w}{ }\PYG{l+s}{plane}\PYG{+w}{ }\PYG{l+s}{1}\PYG{+w}{ }\PYG{l+s}{of}\PYG{+w}{ }\PYG{l+s}{30...}\PYG{+w}{  }\PYG{l+s}{\PYGZpc{}\PYGZpc{}}\PYG{+w}{ }\PYG{l+s}{check}\PYG{+w}{ }\PYG{l+s}{this}\PYG{+w}{ }\PYG{l+s}{matches}\PYG{+w}{ }\PYG{l+s}{the}\PYG{+w}{ }\PYG{l+s}{number}\PYG{+w}{ }\PYG{l+s}{of}\PYG{+w}{ }\PYG{l+s}{Z\PYGZhy{}Planes}\PYG{+w}{ }\PYG{l+s}{you}\PYG{+w}{ }\PYG{l+s}{expect}
\PYG{n}{Data}\PYG{+w}{ }\PYG{l+s}{loaded}\PYG{+w}{ }\PYG{l+s}{in.}\PYG{+w}{ }\PYG{l+s}{This}\PYG{+w}{ }\PYG{l+s}{process}\PYG{+w}{ }\PYG{l+s}{takes}\PYG{+w}{ }\PYG{l+s}{0.024489}\PYG{+w}{ }\PYG{l+s}{minutes.}
\PYG{n}{Beginning}\PYG{+w}{ }\PYG{l+s}{patched,}\PYG{+w}{ }\PYG{l+s}{volumetric}\PYG{+w}{ }\PYG{l+s}{CNMF...}
\end{sphinxVerbatim}


\subsubsection{AtoAc}
\label{\detokenize{user_guide/segmentation:atoac}}
\sphinxAtStartPar
Turn the CaImAn output A (sparse, spatial footprints for entire FOV) into Ac (sparse, spatial footprints localized around each neuron).
\sphinxhyphen{} Standardizes the size of each neuron’s footprint to a uniform (4*tau+1, 4*tau+1) matrix, centered on the neuron’s centroid {[}acx x acy{]}.

\noindent\sphinxincludegraphics{{sparse_rep1}.png}


\subsubsection{Component Validation}
\label{\detokenize{user_guide/segmentation:component-validation}}
\sphinxAtStartPar
The key idea for validating our neurons is that \sphinxstylestrong{we know how long the brightness indicating neurons activity should stay bright} as a function
of the number of frames. That is, our calcium indicator (in this example: GCaMP\sphinxhyphen{}6s), with a rise\sphinxhyphen{}time of 250ms and a decay\sphinxhyphen{}time of 500ms = 750ms, while we
record at 4.7 frames/second = “Samples per transient=round(4.7Hz×(0.2s+0.55s))=3”
\begin{itemize}
\item {} 
\sphinxAtStartPar
Use the decay time (0.5s) multiplied by the number of frames to estimate the number of samples expected in the movie.

\item {} 
\sphinxAtStartPar
Calculate the likelihood of an unexpected event (e.g., a spike) and return a value metric for the quality of the components.
\sphinxhyphen{} Normal Cumulative Distribution function, input = \sphinxhyphen{}min\_SNR.

\item {} 
\sphinxAtStartPar
Evaluate the likelihood of observing traces given the distribution of noise.

\end{itemize}


\subsubsection{Parameters}
\label{\detokenize{user_guide/segmentation:parameters}}
\sphinxAtStartPar
There are many, many parameters used in segmentation and deconvolution. Many of the parameters are sensitive to the  pixel resolution and FOV of the recording. This section discusses such parameters.
The most influencial parameters hold information about the size of neurons and dynamics of the calcium indicator in time.


\paragraph{Tau}
\label{\detokenize{user_guide/segmentation:tau}}\begin{itemize}
\item {} 
\sphinxAtStartPar
Tau is the \sphinxtitleref{half\sphinxhyphen{}size} of a neuron. If a neuron is 10 micron, tau will be a 5 micon.

\item {} 
\sphinxAtStartPar
In general, round up.

\item {} 
\sphinxAtStartPar
The kernel is fixed to have this decay and is not fit to the data.

\end{itemize}


\paragraph{merge\_thresh}
\label{\detokenize{user_guide/segmentation:merge-thresh}}\begin{itemize}
\item {} 
\sphinxAtStartPar
The value of the correlation coefficient (between 0\sphinxhyphen{}1) at which two neurons are considered “the same neuron!”, thus merge them.

\item {} 
\sphinxAtStartPar
This correlation is done temporally.

\end{itemize}


\paragraph{Exact CaImAn Parameters}
\label{\detokenize{user_guide/segmentation:exact-caiman-parameters}}
\begin{sphinxVerbatim}[commandchars=\\\{\}]
\PYG{n}{merge\PYGZus{}thresh}\PYG{+w}{ }\PYG{p}{=}\PYG{+w}{ }\PYG{l+m+mf}{0.8}\PYG{p}{;}
\PYG{n}{min\PYGZus{}SNR}\PYG{+w}{ }\PYG{p}{=}\PYG{+w}{ }\PYG{l+m+mf}{1.4}\PYG{p}{;}\PYG{+w}{ }\PYG{c}{\PYGZpc{} liberal threshold, tighten in post}
\PYG{n}{space\PYGZus{}thresh}\PYG{+w}{ }\PYG{p}{=}\PYG{+w}{ }\PYG{l+m+mf}{0.2}\PYG{p}{;}\PYG{+w}{ }\PYG{c}{\PYGZpc{} threshold for spatial components}
\PYG{n}{time\PYGZus{}thresh}\PYG{+w}{ }\PYG{p}{=}\PYG{+w}{ }\PYG{l+m+mf}{0.0}\PYG{p}{;}
\PYG{n}{sz}\PYG{+w}{ }\PYG{p}{=}\PYG{+w}{ }\PYG{l+m+mf}{0.1}\PYG{p}{;}\PYG{+w}{ }\PYG{c}{\PYGZpc{} IF FOOTPRINTS ARE TOO SMALL, CONSIDER sz = 0.1}
\PYG{n}{mx}\PYG{+w}{ }\PYG{p}{=}\PYG{+w}{ }\PYG{n+nb}{ceil}\PYG{p}{(}\PYG{n+nb}{pi}\PYG{o}{.*}\PYG{p}{(}\PYG{l+m+mf}{1.33}\PYG{o}{.*}\PYG{n}{tau}\PYG{p}{)}\PYG{o}{.\PYGZca{}}\PYG{l+m+mi}{2}\PYG{p}{)}\PYG{p}{;}
\PYG{n}{mn}\PYG{+w}{ }\PYG{p}{=}\PYG{+w}{ }\PYG{n+nb}{floor}\PYG{p}{(}\PYG{n+nb}{pi}\PYG{o}{.*}\PYG{p}{(}\PYG{n}{tau}\PYG{o}{.*}\PYG{l+m+mf}{0.5}\PYG{p}{)}\PYG{o}{.\PYGZca{}}\PYG{l+m+mi}{2}\PYG{p}{)}\PYG{p}{;}\PYG{+w}{ }\PYG{c}{\PYGZpc{} SHRINK IF FOOTPRINTS ARE TOO SMALL}
\PYG{n}{p}\PYG{+w}{ }\PYG{p}{=}\PYG{+w}{ }\PYG{l+m+mi}{2}\PYG{p}{;}\PYG{+w}{ }\PYG{c}{\PYGZpc{} order of dynamics}
\PYG{n}{sizY}\PYG{+w}{ }\PYG{p}{=}\PYG{+w}{ }\PYG{n+nb}{size}\PYG{p}{(}\PYG{n}{data}\PYG{p}{)}\PYG{p}{;}
\PYG{n}{patch\PYGZus{}size}\PYG{+w}{ }\PYG{p}{=}\PYG{+w}{ }\PYG{n+nb}{round}\PYG{p}{(}\PYG{l+m+mi}{650}\PYG{o}{/}\PYG{n}{pixel\PYGZus{}resolution}\PYG{p}{)}\PYG{o}{.*}\PYG{p}{[}\PYG{l+m+mi}{1}\PYG{p}{,}\PYG{l+m+mi}{1}\PYG{p}{]}\PYG{p}{;}
\PYG{n}{overlap}\PYG{+w}{ }\PYG{p}{=}\PYG{+w}{ }\PYG{p}{[}\PYG{l+m+mi}{1}\PYG{p}{,}\PYG{l+m+mi}{1}\PYG{p}{]}\PYG{o}{.*}\PYG{n+nb}{ceil}\PYG{p}{(}\PYG{l+m+mi}{50}\PYG{o}{./}\PYG{n}{pixel\PYGZus{}resolution}\PYG{p}{)}\PYG{p}{;}
\PYG{n}{patches}\PYG{+w}{ }\PYG{p}{=}\PYG{+w}{ }\PYG{n}{construct\PYGZus{}patches}\PYG{p}{(}\PYG{n}{sizY}\PYG{p}{(}\PYG{l+m+mi}{1}\PYG{p}{:}\PYG{k}{end}\PYG{o}{\PYGZhy{}}\PYG{l+m+mi}{1}\PYG{p}{)}\PYG{p}{,}\PYG{n}{patch\PYGZus{}size}\PYG{p}{,}\PYG{n}{overlap}\PYG{p}{)}\PYG{p}{;}
\PYG{c}{\PYGZpc{} number of components based on assumption of 9.2e4 neurons/mm\PYGZca{}3}
\PYG{n}{K}\PYG{+w}{ }\PYG{p}{=}\PYG{+w}{ }\PYG{n+nb}{ceil}\PYG{p}{(}\PYG{l+m+mf}{9.2e4}\PYG{o}{.*}\PYG{l+m+mf}{20e\PYGZhy{}9}\PYG{o}{.*}\PYG{p}{(}\PYG{n}{pixel\PYGZus{}resolution}\PYG{o}{.*}\PYG{n}{patch\PYGZus{}size}\PYG{p}{(}\PYG{l+m+mi}{1}\PYG{p}{)}\PYG{p}{)}\PYG{o}{.\PYGZca{}}\PYG{l+m+mi}{2}\PYG{p}{)}\PYG{p}{;}
\end{sphinxVerbatim}
\begin{itemize}
\item {} 
\sphinxAtStartPar
\sphinxstylestrong{merge\_thresh}: Checking the temporal correlation between components that overlap in space. If they have at least 1px in common and the correlation is above the merge threshold, merge the components.

\item {} 
\sphinxAtStartPar
Factorization via CNMF yields “raw” traces (“y”). These raw traces are noisy and jagged.

\item {} 
\sphinxAtStartPar
Each raw trace is deconvolved via “constrained foopsi,” which yields the decay (and for p=2, rise) coefficients (“g”) and the vector of “spiking” activity (“S”) that best explain the raw trace. S should ideally be \textasciitilde{}90\% zeros.

\item {} 
\sphinxAtStartPar
S and g are then used to produce C, which (hopefully) looks like the raw trace Y, but much cleaner and smoother. The optional output YrA is equal to Y\sphinxhyphen{}C, representing the original raw trace.

\end{itemize}


\subsubsection{Deconvolution}
\label{\detokenize{user_guide/segmentation:deconvolution}}
\sphinxAtStartPar
TODO: put this foopsi trickyness information in “For Developers” section

\sphinxAtStartPar
FOOPSI (Fast OOPSI) is originally from “Fast Nonnegative Deconvolution for Spike Train Inference From Population Calcium Imaging” by Vogelstein et al. (2010).
\sphinxhyphen{} OASIS was introduced in “Fast Active Set Methods for Online Spike Inference from Calcium Imaging” by Friedrich \& Paninski (2016).
\sphinxhyphen{} Most of the CAIMAN\sphinxhyphen{}MATLAB code uses OASIS, not FOOPSI, despite some functions being named “foopsi\_oasis.”

\sphinxAtStartPar
Branches from the main “deconvolveCa” function in MATLAB\_CAIMAN:

\sphinxAtStartPar
\sphinxstylestrong{oasis} branches: Despite some being named “foopsi\_oasis,” they use OASIS math.
\sphinxhyphen{} foopsi\_oasisAR1
\sphinxhyphen{} foopsi\_oasisAR2
\sphinxhyphen{} constrained\_oasisAR1
\sphinxhyphen{} thresholded\_oasisAR1
\sphinxhyphen{} thresholded\_oasisAR2
\sphinxstylestrong{constrained\_foopsi} branch: Used if method=”constrained” and model type is not “ar1” (e.g., ar2).
\sphinxhyphen{} Optimization methods: CVX (external), SPGL1 (external), LARS, dual.
\sphinxstylestrong{onnls} branch: Used if method=”foopsi” or “thresholded” with model type=”exp2” or “kernel.” Based on OASIS.

\sphinxstepscope


\subsection{Axial\sphinxhyphen{}Offset Correction}
\label{\detokenize{user_guide/offset_correction:axial-offset-correction}}\label{\detokenize{user_guide/offset_correction:offset-correction}}\label{\detokenize{user_guide/offset_correction::doc}}
\sphinxAtStartPar
Before proceeding:
\begin{itemize}
\item {} 
\sphinxAtStartPar
You will need to be in a GUI environment for this step. Calculate offset will show you two images, click the feature that matches in both images.

\item {} 
\sphinxAtStartPar
You will need the following calibration files:
\begin{itemize}
\item {} 
\sphinxAtStartPar
\sphinxtitleref{pollen\_calibration\_Z\_vs\_N.fig}

\item {} 
\sphinxAtStartPar
\sphinxtitleref{pllen\_calibration\_x\_y\_offsets.fig}

\end{itemize}

\end{itemize}

\sphinxAtStartPar
Place these files in the same directory as your \sphinxtitleref{caiman\_output\_plane\_N} files.

\begin{sphinxVerbatim}[commandchars=\\\{\}]
\PYG{+w}{      }\PYG{o}{\PYGZgt{}}\PYG{o}{\PYGZgt{}}\PYG{+w}{ }\PYG{n+nb}{help}\PYG{+w}{ }\PYG{n}{collatePlanes}
\PYG{n}{collatePlanes}\PYG{+w}{ }\PYG{l+s}{Analyzes}\PYG{+w}{ }\PYG{l+s}{and}\PYG{+w}{ }\PYG{l+s}{processes}\PYG{+w}{ }\PYG{l+s}{imaging}\PYG{+w}{ }\PYG{l+s}{data}\PYG{+w}{ }\PYG{l+s}{by}\PYG{+w}{ }\PYG{l+s}{extracting}\PYG{+w}{ }\PYG{l+s}{and}\PYG{+w}{ }\PYG{l+s}{correcting}\PYG{+w}{ }\PYG{l+s}{features}\PYG{+w}{ }\PYG{l+s}{across}\PYG{+w}{ }\PYG{l+s}{multiple}\PYG{+w}{ }\PYG{l+s}{planes.}

\PYG{n}{This}\PYG{+w}{ }\PYG{l+s}{function}\PYG{+w}{ }\PYG{l+s}{analyzes}\PYG{+w}{ }\PYG{l+s}{imaging}\PYG{+w}{ }\PYG{l+s}{data}\PYG{+w}{ }\PYG{l+s}{from}\PYG{+w}{ }\PYG{l+s}{a}\PYG{+w}{ }\PYG{l+s}{specified}\PYG{+w}{ }\PYG{l+s}{directory,}\PYG{+w}{ }\PYG{l+s}{applying}
\PYG{n}{various}\PYG{+w}{ }\PYG{l+s}{thresholds}\PYG{+w}{ }\PYG{l+s}{and}\PYG{+w}{ }\PYG{l+s}{corrections}\PYG{+w}{ }\PYG{l+s}{based}\PYG{+w}{ }\PYG{l+s}{on}\PYG{+w}{ }\PYG{l+s}{metadata.}\PYG{+w}{ }\PYG{l+s}{It}\PYG{+w}{ }\PYG{l+s}{processes}\PYG{+w}{ }\PYG{l+s}{neuron}
\PYG{n}{activity}\PYG{+w}{ }\PYG{l+s}{data,}\PYG{+w}{ }\PYG{l+s}{handles}\PYG{+w}{ }\PYG{l+s}{z\PYGZhy{}plane}\PYG{+w}{ }\PYG{l+s}{corrections,}\PYG{+w}{ }\PYG{l+s}{and}\PYG{+w}{ }\PYG{l+s}{outputs}\PYG{+w}{ }\PYG{l+s}{figures}\PYG{+w}{ }\PYG{l+s}{representing}
\PYG{n}{neuron}\PYG{+w}{ }\PYG{l+s}{distributions}\PYG{+w}{ }\PYG{l+s}{along}\PYG{+w}{ }\PYG{l+s}{with}\PYG{+w}{ }\PYG{l+s}{collated}\PYG{+w}{ }\PYG{l+s}{data}\PYG{+w}{ }\PYG{l+s}{files.}

\PYG{n}{The}\PYG{+w}{ }\PYG{l+s}{function}\PYG{+w}{ }\PYG{l+s}{expects}\PYG{+w}{ }\PYG{l+s}{the}\PYG{+w}{ }\PYG{l+s}{directory}\PYG{+w}{ }\PYG{l+s}{to}\PYG{+w}{ }\PYG{l+s}{contain}\PYG{+w}{ }\PYG{l+s}{\PYGZsq{}caiman\PYGZus{}output\PYGZus{}plane\PYGZus{}*.mat\PYGZsq{}}\PYG{+w}{ }\PYG{l+s}{files}
\PYG{n}{with}\PYG{+w}{ }\PYG{l+s}{variables}\PYG{+w}{ }\PYG{l+s}{related}\PYG{+w}{ }\PYG{l+s}{to}\PYG{+w}{ }\PYG{l+s}{neuronal}\PYG{+w}{ }\PYG{l+s}{activity,}\PYG{+w}{ }\PYG{l+s}{and}\PYG{+w}{ }\PYG{l+s}{uses}\PYG{+w}{ }\PYG{l+s}{provided}\PYG{+w}{ }\PYG{l+s}{metadata}\PYG{+w}{ }\PYG{l+s}{for}
\PYG{n}{processing}\PYG{+w}{ }\PYG{l+s}{parameters.}\PYG{+w}{ }\PYG{l+s}{It}\PYG{+w}{ }\PYG{l+s}{adjusts}\PYG{+w}{ }\PYG{l+s}{parameters}\PYG{+w}{ }\PYG{l+s}{dynamically}\PYG{+w}{ }\PYG{l+s}{based}\PYG{+w}{ }\PYG{l+s}{on}\PYG{+w}{ }\PYG{l+s}{the}\PYG{+w}{ }\PYG{l+s}{content}
\PYG{n}{of}\PYG{+w}{ }\PYG{l+s}{metadata}\PYG{+w}{ }\PYG{l+s}{and}\PYG{+w}{ }\PYG{l+s}{filters,}\PYG{+w}{ }\PYG{l+s}{merges}\PYG{+w}{ }\PYG{l+s}{data}\PYG{+w}{ }\PYG{l+s}{across}\PYG{+w}{ }\PYG{l+s}{imaging}\PYG{+w}{ }\PYG{l+s}{planes,}\PYG{+w}{ }\PYG{l+s}{and}\PYG{+w}{ }\PYG{l+s}{performs}
\PYG{n}{z}\PYG{o}{\PYGZhy{}}\PYG{n}{plane}\PYG{+w}{ }\PYG{n}{and}\PYG{+w}{ }\PYG{n}{field}\PYG{+w}{ }\PYG{n}{curvature}\PYG{+w}{ }\PYG{n}{corrections}\PYG{p}{.}

\PYG{n}{Parameters}
\PYG{o}{\PYGZhy{}}\PYG{o}{\PYGZhy{}}\PYG{o}{\PYGZhy{}}\PYG{o}{\PYGZhy{}}\PYG{o}{\PYGZhy{}}\PYG{o}{\PYGZhy{}}\PYG{o}{\PYGZhy{}}\PYG{o}{\PYGZhy{}}\PYG{o}{\PYGZhy{}}\PYG{o}{\PYGZhy{}}
\PYG{n}{dataPath}\PYG{+w}{ }\PYG{l+s}{:}\PYG{+w}{ }\PYG{l+s}{string}
\PYG{+w}{    }\PYG{n}{Path}\PYG{+w}{ }\PYG{n}{to}\PYG{+w}{ }\PYG{n}{the}\PYG{+w}{ }\PYG{n}{directory}\PYG{+w}{ }\PYG{n}{containing}\PYG{+w}{ }\PYG{n}{the}\PYG{+w}{ }\PYG{n}{data}\PYG{+w}{ }\PYG{n}{files}\PYG{+w}{ }\PYG{k}{for}\PYG{+w}{ }\PYG{n}{analysis}\PYG{p}{.}
\PYG{n}{data}\PYG{+w}{ }\PYG{l+s}{:}\PYG{+w}{ }\PYG{l+s}{string}\PYG{+w}{ }\PYG{l+s}{(unused,}\PYG{+w}{ }\PYG{l+s}{placeholder}\PYG{+w}{ }\PYG{l+s}{for}\PYG{+w}{ }\PYG{l+s}{future}\PYG{+w}{ }\PYG{l+s}{use)}
\PYG{+w}{    }\PYG{n}{Placeholder}\PYG{+w}{ }\PYG{n}{parameter}\PYG{+w}{ }\PYG{k}{for}\PYG{+w}{ }\PYG{n}{passing}\PYG{+w}{ }\PYG{n}{data}\PYG{+w}{ }\PYG{n}{directly}\PYG{+w}{ }\PYG{k}{if}\PYG{+w}{ }\PYG{n}{needed}\PYG{p}{.}
\PYG{n}{metadata}\PYG{+w}{ }\PYG{l+s}{:}\PYG{+w}{ }\PYG{l+s}{struct}
\PYG{+w}{    }\PYG{n}{Structure}\PYG{+w}{ }\PYG{n}{containing}\PYG{+w}{ }\PYG{n}{metadata}\PYG{+w}{ }\PYG{k}{for}\PYG{+w}{ }\PYG{n}{processing}\PYG{p}{.}\PYG{+w}{ }\PYG{n}{Must}\PYG{+w}{ }\PYG{n}{include}\PYG{+w}{ }\PYG{n}{fields}\PYG{p}{:}
\PYG{+w}{    }\PYG{n}{r\PYGZus{}thr}\PYG{p}{,}\PYG{+w}{ }\PYG{n}{pixel\PYGZus{}resolution}\PYG{p}{,}\PYG{+w}{ }\PYG{n}{min\PYGZus{}snr}\PYG{p}{,}\PYG{+w}{ }\PYG{n}{frame\PYGZus{}rate}\PYG{p}{,}\PYG{+w}{ }\PYG{n}{fovx}\PYG{p}{,}\PYG{+w}{ }\PYG{n}{and}\PYG{+w}{ }\PYG{n}{fovy}\PYG{p}{.}
\PYG{n}{startDepth}\PYG{+w}{ }\PYG{l+s}{:}\PYG{+w}{ }\PYG{l+s}{double}
\PYG{+w}{    }\PYG{n}{The}\PYG{+w}{ }\PYG{n}{starting}\PYG{+w}{ }\PYG{n}{depth}\PYG{+w}{ }\PYG{p}{(}\PYG{n}{z0}\PYG{p}{)}\PYG{+w}{ }\PYG{n}{from}\PYG{+w}{ }\PYG{n+nb}{which}\PYG{+w}{ }\PYG{n}{processing}\PYG{+w}{ }\PYG{n}{should}\PYG{+w}{ }\PYG{n}{begin}\PYG{p}{;}\PYG{+w}{ }\PYG{k}{if}\PYG{+w}{ }\PYG{n}{not}
\PYG{+w}{    }\PYG{n}{provided}\PYG{p}{,}\PYG{+w}{ }\PYG{n}{a}\PYG{+w}{ }\PYG{n+nb}{dialog}\PYG{+w}{ }\PYG{n}{will}\PYG{+w}{ }\PYG{n}{prompt}\PYG{+w}{ }\PYG{k}{for}\PYG{+w}{ }\PYG{n+nb}{input}\PYG{p}{.}

\PYG{n}{Returns}
\PYG{o}{\PYGZhy{}}\PYG{o}{\PYGZhy{}}\PYG{o}{\PYGZhy{}}\PYG{o}{\PYGZhy{}}\PYG{o}{\PYGZhy{}}\PYG{o}{\PYGZhy{}}\PYG{o}{\PYGZhy{}}
\PYG{n}{None}

\PYG{l+s}{Outputs}
\PYG{o}{\PYGZhy{}}\PYG{o}{\PYGZhy{}}\PYG{o}{\PYGZhy{}}\PYG{o}{\PYGZhy{}}\PYG{o}{\PYGZhy{}}\PYG{o}{\PYGZhy{}}\PYG{o}{\PYGZhy{}}
\PYG{o}{\PYGZhy{}}\PYG{+w}{ }\PYG{p}{.}\PYG{n}{fig}\PYG{+w}{ }\PYG{n}{files}\PYG{+w}{ }\PYG{n}{showing}\PYG{+w}{ }\PYG{n}{neuron}\PYG{+w}{ }\PYG{n}{distributions}\PYG{+w}{ }\PYG{n}{in}\PYG{+w}{ }\PYG{n}{z}\PYG{+w}{ }\PYG{n}{and}\PYG{+w}{ }\PYG{n}{radial}\PYG{+w}{ }\PYG{n}{directions}\PYG{p}{.}
\PYG{o}{\PYGZhy{}}\PYG{+w}{ }\PYG{n}{A}\PYG{+w}{ }\PYG{p}{.}\PYG{n}{mat}\PYG{+w}{ }\PYG{n}{file}\PYG{+w}{ }\PYG{n}{with}\PYG{+w}{ }\PYG{n}{collated}\PYG{+w}{ }\PYG{n}{and}\PYG{+w}{ }\PYG{n}{processed}\PYG{+w}{ }\PYG{n}{imaging}\PYG{+w}{ }\PYG{n}{data}\PYG{p}{.}

\PYG{n}{Notes}
\PYG{o}{\PYGZhy{}}\PYG{o}{\PYGZhy{}}\PYG{o}{\PYGZhy{}}\PYG{o}{\PYGZhy{}}\PYG{o}{\PYGZhy{}}
\PYG{o}{\PYGZhy{}}\PYG{+w}{ }\PYG{n}{Expects}\PYG{+w}{ }\PYG{l+s}{\PYGZsq{}}\PYG{l+s}{three\PYGZus{}neuron\PYGZus{}mean\PYGZus{}offsets.mat\PYGZsq{}}\PYG{+w}{ }\PYG{n}{and}\PYG{+w}{ }\PYG{l+s}{\PYGZsq{}}\PYG{l+s}{pollen\PYGZus{}calibration\PYGZus{}Z\PYGZus{}vs\PYGZus{}N.fig\PYGZsq{}}
\PYG{+w}{  }\PYG{n}{within}\PYG{+w}{ }\PYG{n}{the}\PYG{+w}{ }\PYG{n}{dataPath}\PYG{+w}{ }\PYG{k}{for}\PYG{+w}{ }\PYG{n}{processing}\PYG{p}{.}
\PYG{o}{\PYGZhy{}}\PYG{+w}{ }\PYG{n}{The}\PYG{+w}{ }\PYG{k}{function}\PYG{+w}{ }\PYG{n}{uses}\PYG{+w}{ }\PYG{n}{parallel}\PYG{+w}{ }\PYG{n}{processing}\PYG{+w}{ }\PYG{k}{for}\PYG{+w}{ }\PYG{n}{some}\PYG{+w}{ }\PYG{n}{calculations}\PYG{+w}{ }\PYG{n}{to}\PYG{+w}{ }\PYG{n}{improve}
\PYG{+w}{  }\PYG{n}{performance}\PYG{p}{.}

\PYG{n}{Examples}
\PYG{o}{\PYGZhy{}}\PYG{o}{\PYGZhy{}}\PYG{o}{\PYGZhy{}}\PYG{o}{\PYGZhy{}}\PYG{o}{\PYGZhy{}}\PYG{o}{\PYGZhy{}}\PYG{o}{\PYGZhy{}}\PYG{o}{\PYGZhy{}}
\PYG{n}{collatePlanes}\PYG{p}{(}\PYG{l+s}{\PYGZsq{}}\PYG{l+s}{C:/data/images/\PYGZsq{}}\PYG{p}{,}\PYG{+w}{ }\PYG{l+s}{\PYGZsq{}}\PYG{l+s}{\PYGZsq{}}\PYG{p}{,}\PYG{+w}{ }\PYG{n+nb}{struct}\PYG{p}{(}\PYG{l+s}{\PYGZsq{}}\PYG{l+s}{r\PYGZus{}thr\PYGZsq{}}\PYG{p}{:}\PYG{l+m+mf}{0.4}\PYG{p}{,}\PYG{+w}{ }\PYG{l+s}{\PYGZsq{}}\PYG{l+s}{pixel\PYGZus{}resolution\PYGZsq{}}\PYG{p}{:}\PYG{l+m+mi}{2}\PYG{p}{,}\PYG{+w}{ }\PYG{l+s}{\PYGZsq{}}\PYG{l+s}{min\PYGZus{}snr\PYGZsq{}}\PYG{p}{:}\PYG{l+m+mf}{1.5}\PYG{p}{,}\PYG{+w}{ }\PYG{l+s}{\PYGZsq{}}\PYG{l+s}{frame\PYGZus{}rate\PYGZsq{}}\PYG{p}{:}\PYG{l+m+mf}{9.61}\PYG{p}{,}\PYG{+w}{ }\PYG{l+s}{\PYGZsq{}}\PYG{l+s}{fovx\PYGZsq{}}\PYG{p}{:}\PYG{l+m+mi}{1200}\PYG{p}{,}\PYG{+w}{ }\PYG{l+s}{\PYGZsq{}}\PYG{l+s}{fovy\PYGZsq{}}\PYG{p}{:}\PYG{l+m+mi}{1164}\PYG{p}{)}\PYG{p}{,}\PYG{+w}{ }\PYG{l+m+mi}{100}\PYG{p}{)}\PYG{p}{;}
\PYG{+w}{  }\PYG{n}{This}\PYG{+w}{ }\PYG{l+s}{example}\PYG{+w}{ }\PYG{l+s}{processes}\PYG{+w}{ }\PYG{l+s}{data}\PYG{+w}{ }\PYG{l+s}{from}\PYG{+w}{ }\PYG{l+s}{\PYGZsq{}C:/data/images/\PYGZsq{}}\PYG{l+s}{,}\PYG{+w}{ }\PYG{l+s}{starting}\PYG{+w}{ }\PYG{l+s}{at}\PYG{+w}{ }\PYG{l+s}{a}\PYG{+w}{ }\PYG{l+s}{depth}\PYG{+w}{ }\PYG{l+s}{of}\PYG{+w}{ }\PYG{l+s}{100}\PYG{+w}{ }\PYG{l+s}{microns,}
\PYG{+w}{  }\PYG{n}{with}\PYG{+w}{ }\PYG{n}{specified}\PYG{+w}{ }\PYG{n}{metadata}\PYG{+w}{ }\PYG{n}{parameters}\PYG{p}{.}

\PYG{n}{See}\PYG{+w}{ }\PYG{l+s}{also}\PYG{+w}{ }\PYG{l+s}{load,}\PYG{+w}{ }\PYG{l+s}{inputdlg,}\PYG{+w}{ }\PYG{l+s}{struct,}\PYG{+w}{ }\PYG{l+s}{fullfile,}\PYG{+w}{ }\PYG{l+s}{exist}
\end{sphinxVerbatim}

\sphinxAtStartPar
The user will be prompted to select the same \sphinxstylestrong{feature} / \sphinxstylestrong{region\sphinxhyphen{}of\sphinxhyphen{}interest} / \sphinxstylestrong{neuron}:

\noindent\sphinxincludegraphics{{compare_planes1}.png}

\sphinxAtStartPar
After selecting 3 neurons for each plane, you are done with the LBM pipeline.

\sphinxstepscope


\subsection{ScanImage Metadata}
\label{\detokenize{user_guide/metadata:scanimage-metadata}}\label{\detokenize{user_guide/metadata::doc}}
\sphinxAtStartPar
As discussed in {\hyperref[\detokenize{user_guide/pre_processing:pipeline-step-1}]{\sphinxcrossref{\DUrole{std,std-ref}{1. Re\sphinxhyphen{}Construct Volumetric Time\sphinxhyphen{}Series}}}}, reconstructing an image from the ScanImage \sphinxtitleref{.tiff} file is handled internally by the pipeline. This describes
the metadata that hold all of the values used for that process.

\sphinxAtStartPar
The funcion {\hyperref[\detokenize{api/utils:get_metadata}]{\sphinxcrossref{\sphinxcode{\sphinxupquote{get\_metadata()}}}}} takes as input a path to any \sphinxhref{https://www.mbfbioscience.com/products/scanimage/}{ScanImage} tiff file and returns the metadata.

\phantomsection\label{\detokenize{user_guide/metadata:mdetadata-disclaimer}}
\begin{sphinxadmonition}{note}{Note:}
\sphinxAtStartPar
During aquisition, the user is prompted to split the recorded frames across multiple .tiff files.
It doesn’t matter which .tiff file is used for this function, the metadata used applies to each file.
\end{sphinxadmonition}

\def\sphinxLiteralBlockLabel{\label{\detokenize{user_guide/metadata:metadata-code}}}
\begin{sphinxVerbatim}[commandchars=\\\{\}]
\PYG{o}{\PYGZgt{}}\PYG{o}{\PYGZgt{}}\PYG{+w}{ }\PYG{n}{get\PYGZus{}metadata}\PYG{p}{(}\PYG{n+nb}{fullfile}\PYG{p}{(}\PYG{n}{extract\PYGZus{}path}\PYG{p}{,}\PYG{+w}{ }\PYG{l+s}{\PYGZdq{}MH184\PYGZus{}both\PYGZus{}6mm\PYGZus{}FOV\PYGZus{}150\PYGZus{}600um\PYGZus{}depth\PYGZus{}410mW\PYGZus{}9min\PYGZus{}no\PYGZus{}stimuli\PYGZus{}00001\PYGZus{}00001.tiff\PYGZdq{}}\PYG{p}{)}\PYG{p}{)}

\PYG{+w}{ }\PYG{n+nb}{ans}\PYG{+w}{ }\PYG{p}{=}

\PYG{+w}{   }\PYG{n+nb}{struct}\PYG{+w}{ }\PYG{n}{with}\PYG{+w}{ }\PYG{n}{fields}\PYG{p}{:}

\PYG{+w}{                        }\PYG{n}{center\PYGZus{}xy}\PYG{p}{:}\PYG{+w}{ }\PYG{p}{[}\PYG{o}{\PYGZhy{}}\PYG{l+m+mf}{15.2381}\PYG{+w}{ }\PYG{l+m+mi}{0}\PYG{p}{]}
\PYG{+w}{                          }\PYG{n}{size\PYGZus{}xy}\PYG{p}{:}\PYG{+w}{ }\PYG{p}{[}\PYG{l+m+mf}{3.8095}\PYG{+w}{ }\PYG{l+m+mf}{38.0952}\PYG{p}{]}
\PYG{+w}{                     }\PYG{n}{num\PYGZus{}pixel\PYGZus{}xy}\PYG{p}{:}\PYG{+w}{ }\PYG{p}{[}\PYG{l+m+mi}{144}\PYG{+w}{ }\PYG{l+m+mi}{1200}\PYG{p}{]}
\PYG{+w}{                  }\PYG{n}{lines\PYGZus{}per\PYGZus{}frame}\PYG{p}{:}\PYG{+w}{ }\PYG{l+m+mi}{144}
\PYG{+w}{                  }\PYG{n}{pixels\PYGZus{}per\PYGZus{}line}\PYG{p}{:}\PYG{+w}{ }\PYG{l+m+mi}{128}
\PYG{+w}{     }\PYG{n}{num\PYGZus{}lines\PYGZus{}between\PYGZus{}scanfields}\PYG{p}{:}\PYG{+w}{ }\PYG{l+m+mi}{24}
\PYG{+w}{                     }\PYG{n}{image\PYGZus{}length}\PYG{p}{:}\PYG{+w}{ }\PYG{l+m+mi}{11008}
\PYG{+w}{                      }\PYG{n}{image\PYGZus{}width}\PYG{p}{:}\PYG{+w}{ }\PYG{l+m+mi}{145}
\PYG{+w}{                }\PYG{n}{full\PYGZus{}image\PYGZus{}height}\PYG{p}{:}\PYG{+w}{ }\PYG{l+m+mi}{1165}
\PYG{+w}{                 }\PYG{n}{full\PYGZus{}image\PYGZus{}width}\PYG{p}{:}\PYG{+w}{ }\PYG{l+m+mi}{1197}
\PYG{+w}{                       }\PYG{n}{num\PYGZus{}planes}\PYG{p}{:}\PYG{+w}{ }\PYG{l+m+mi}{30}
\PYG{+w}{                         }\PYG{n}{num\PYGZus{}rois}\PYG{p}{:}\PYG{+w}{ }\PYG{l+m+mi}{9}
\PYG{+w}{                 }\PYG{n}{num\PYGZus{}frames\PYGZus{}total}\PYG{p}{:}\PYG{+w}{ }\PYG{l+m+mi}{1176}
\PYG{+w}{                  }\PYG{n}{num\PYGZus{}frames\PYGZus{}file}\PYG{p}{:}\PYG{+w}{ }\PYG{l+m+mi}{392}
\PYG{+w}{                        }\PYG{n}{num\PYGZus{}files}\PYG{p}{:}\PYG{+w}{ }\PYG{l+m+mi}{3}
\PYG{+w}{                       }\PYG{n}{frame\PYGZus{}rate}\PYG{p}{:}\PYG{+w}{ }\PYG{l+m+mf}{2.1797}
\PYG{+w}{             }\PYG{n}{objective\PYGZus{}resolution}\PYG{p}{:}\PYG{+w}{ }\PYG{l+m+mf}{157.5000}
\PYG{+w}{                              }\PYG{n}{fov}\PYG{p}{:}\PYG{+w}{ }\PYG{p}{[}\PYG{l+m+mi}{600}\PYG{+w}{ }\PYG{l+m+mi}{6000}\PYG{p}{]}
\PYG{+w}{                }\PYG{n}{strip\PYGZus{}width\PYGZus{}slice}\PYG{p}{:}\PYG{+w}{ }\PYG{p}{[}\PYG{l+m+mi}{8}\PYG{+w}{ }\PYG{l+m+mi}{9}\PYG{+w}{ }\PYG{l+m+mi}{10}\PYG{+w}{ }\PYG{l+m+mi}{11}\PYG{+w}{ }\PYG{l+m+mi}{12}\PYG{+w}{ }\PYG{l+m+mi}{13}\PYG{+w}{ }\PYG{l+m+mi}{14}\PYG{+w}{ }\PYG{l+m+mi}{15}\PYG{+w}{ }\PYG{l+m+mi}{16}\PYG{+w}{ }\PYG{l+m+mi}{17}\PYG{+w}{ }\PYG{l+m+mi}{18}\PYG{+w}{ }\PYG{l+m+mi}{19}\PYG{+w}{ }\PYG{l+m+mi}{20}\PYG{+w}{ }\PYG{l+m+mi}{21}\PYG{+w}{ }\PYG{l+m+mi}{22}\PYG{+w}{ }\PYG{l+m+mi}{23}\PYG{+w}{ }\PYG{l+m+mi}{24}\PYG{+w}{ }\PYG{l+m+mi}{25}\PYG{+w}{ }\PYG{l+m+mi}{26}\PYG{+w}{ }\PYG{l+m+mi}{27}\PYG{+w}{ }\PYG{l+m+mi}{28}\PYG{+w}{ }\PYG{l+m+mi}{29}\PYG{+w}{ }\PYG{l+m+mi}{30}\PYG{+w}{ }\PYG{l+m+mi}{31}\PYG{+w}{ }\PYG{l+m+mi}{32}\PYG{+w}{ }…\PYG{+w}{ }\PYG{p}{]}\PYG{+w}{ }\PYG{p}{(}\PYG{l+m+mi}{1}×\PYG{l+m+mi}{129}\PYG{+w}{ }\PYG{n+nb}{double}\PYG{p}{)}
\PYG{+w}{                      }\PYG{n}{strip\PYGZus{}width}\PYG{p}{:}\PYG{+w}{ }\PYG{l+m+mi}{129}
\PYG{+w}{                 }\PYG{n}{pixel\PYGZus{}resolution}\PYG{p}{:}\PYG{+w}{ }\PYG{l+m+mf}{4.5833}
\PYG{+w}{                    }\PYG{n}{sample\PYGZus{}format}\PYG{p}{:}\PYG{+w}{ }\PYG{l+s}{\PYGZsq{}}\PYG{l+s}{int16\PYGZsq{}}
\PYG{+w}{                   }\PYG{n}{extra\PYGZus{}width\PYGZus{}px}\PYG{p}{:}\PYG{+w}{ }\PYG{l+m+mi}{16}
\PYG{+w}{          }\PYG{n}{extra\PYGZus{}width\PYGZus{}per\PYGZus{}side\PYGZus{}px}\PYG{p}{:}\PYG{+w}{ }\PYG{l+m+mi}{8}
\PYG{+w}{                    }\PYG{n}{base\PYGZus{}filename}\PYG{p}{:}\PYG{+w}{ }\PYG{l+s}{\PYGZdq{}MH184\PYGZus{}both\PYGZus{}6mm\PYGZus{}FOV\PYGZus{}150\PYGZus{}600um\PYGZus{}depth\PYGZus{}410mW\PYGZus{}9min\PYGZus{}no\PYGZus{}stimuli\PYGZus{}00001\PYGZus{}00001\PYGZdq{}}
\PYG{+w}{                    }\PYG{n}{base\PYGZus{}filepath}\PYG{p}{:}\PYG{+w}{ }\PYG{l+s}{\PYGZdq{}\PYGZbs{}raw\PYGZdq{}}
\PYG{+w}{                     }\PYG{n}{base\PYGZus{}fileext}\PYG{p}{:}\PYG{+w}{ }\PYG{l+s}{\PYGZdq{}.tif\PYGZdq{}}
\end{sphinxVerbatim}

\sphinxstepscope


\subsection{Tips and Tricks}
\label{\detokenize{user_guide/tips_tricks:tips-and-tricks}}\label{\detokenize{user_guide/tips_tricks::doc}}

\subsubsection{Learn about Functions}
\label{\detokenize{user_guide/tips_tricks:learn-about-functions}}\label{\detokenize{user_guide/tips_tricks:help-functions}}
\begin{DUlineblock}{0em}
\item[] Run ‘help \textless{}function\textgreater{}’ in the command window for a detailed overview on function parameters, outputs and examples.
\end{DUlineblock}

\begin{sphinxVerbatim}[commandchars=\\\{\}]
\PYG{o}{\PYGZgt{}}\PYG{o}{\PYGZgt{}}\PYG{+w}{ }\PYG{n+nb}{help}\PYG{+w}{ }\PYG{n}{convertScanImageTiffToVolume}
\PYG{+w}{  }\PYG{n}{convertScanImageTiffToVolume}\PYG{+w}{ }\PYG{n}{Convert}\PYG{+w}{ }\PYG{n}{ScanImage}\PYG{+w}{ }\PYG{p}{.}\PYG{n}{tif}\PYG{+w}{ }\PYG{n}{files}\PYG{+w}{ }\PYG{n}{into}\PYG{+w}{ }\PYG{n}{a}\PYG{+w}{ }4\PYG{n}{D}\PYG{+w}{ }\PYG{n+nb}{volume}\PYG{p}{.}

\PYG{+w}{   }\PYG{n}{Convert}\PYG{+w}{ }\PYG{n}{raw}\PYG{+w}{ }`\PYG{n}{ScanImage}`\PYG{n}{\PYGZus{}}\PYG{+w}{ }\PYG{n}{multi}\PYG{o}{\PYGZhy{}}\PYG{n}{roi}\PYG{+w}{ }\PYG{p}{.}\PYG{n}{tif}\PYG{+w}{ }\PYG{n}{files}\PYG{+w}{ }\PYG{n}{from}\PYG{+w}{ }\PYG{n}{a}\PYG{+w}{ }\PYG{n+nb}{single}\PYG{+w}{ }\PYG{n}{session}
\PYG{+w}{   }\PYG{n}{into}\PYG{+w}{ }\PYG{n}{a}\PYG{+w}{ }\PYG{n+nb}{single}\PYG{+w}{ }4\PYG{n}{D}\PYG{+w}{ }\PYG{n+nb}{volume}\PYG{+w}{ }\PYG{p}{(}\PYG{n}{x}\PYG{p}{,}\PYG{+w}{ }\PYG{n}{y}\PYG{p}{,}\PYG{+w}{ }\PYG{n}{z}\PYG{p}{,}\PYG{+w}{ }\PYG{n}{t}\PYG{p}{)}\PYG{p}{.}\PYG{+w}{ }\PYG{n}{It}\PYG{o}{\PYGZsq{}}\PYG{n}{s}\PYG{+w}{ }\PYG{n}{designed}\PYG{+w}{ }\PYG{n}{to}\PYG{+w}{ }\PYG{n}{process}\PYG{+w}{ }\PYG{n}{files}\PYG{+w}{ }\PYG{k}{for}\PYG{+w}{ }\PYG{n}{the}
\PYG{+w}{   }\PYG{n}{ScanImage}\PYG{+w}{ }\PYG{n}{Version}\PYG{p}{:}\PYG{+w}{ }\PYG{l+m+mi}{2016}\PYG{+w}{ }\PYG{n}{software}\PYG{p}{.}

\PYG{+w}{   }\PYG{n}{Parameters}
\PYG{+w}{   }\PYG{o}{\PYGZhy{}}\PYG{o}{\PYGZhy{}}\PYG{o}{\PYGZhy{}}\PYG{o}{\PYGZhy{}}\PYG{o}{\PYGZhy{}}\PYG{o}{\PYGZhy{}}\PYG{o}{\PYGZhy{}}\PYG{o}{\PYGZhy{}}\PYG{o}{\PYGZhy{}}\PYG{o}{\PYGZhy{}}
\PYG{+w}{   }\PYG{n}{filePath}\PYG{+w}{ }\PYG{p}{:}\PYG{+w}{ }\PYG{n+nb}{char}
\PYG{+w}{       }\PYG{n}{The}\PYG{+w}{ }\PYG{n}{directory}\PYG{+w}{ }\PYG{n}{containing}\PYG{+w}{ }\PYG{n}{the}\PYG{+w}{ }\PYG{n}{raw}\PYG{+w}{ }\PYG{p}{.}\PYG{n}{tif}\PYG{+w}{ }\PYG{n}{files}\PYG{p}{.}\PYG{+w}{ }\PYG{n}{Only}\PYG{+w}{ }\PYG{n}{raw}\PYG{+w}{ }\PYG{p}{.}\PYG{n}{tif}\PYG{+w}{ }\PYG{n}{files}\PYG{+w}{ }\PYG{n}{from}\PYG{+w}{ }\PYG{n}{one}
\PYG{+w}{       }\PYG{n}{session}\PYG{+w}{ }\PYG{n}{should}\PYG{+w}{ }\PYG{n}{be}\PYG{+w}{ }\PYG{n}{in}\PYG{+w}{ }\PYG{n}{the}\PYG{+w}{ }\PYG{n}{directory}\PYG{p}{.}
\PYG{+w}{   }\PYG{n}{saveDirPath}\PYG{+w}{ }\PYG{p}{:}\PYG{+w}{ }\PYG{n+nb}{char}\PYG{p}{,}\PYG{+w}{ }\PYG{n}{optional}
\PYG{+w}{       }\PYG{n}{The}\PYG{+w}{ }\PYG{n}{directory}\PYG{+w}{ }\PYG{n}{where}\PYG{+w}{ }\PYG{n}{processed}\PYG{+w}{ }\PYG{n}{files}\PYG{+w}{ }\PYG{n}{will}\PYG{+w}{ }\PYG{n}{be}\PYG{+w}{ }\PYG{n}{saved}\PYG{p}{.}\PYG{+w}{ }\PYG{n}{It}\PYG{+w}{ }\PYG{n}{is}\PYG{+w}{ }\PYG{n}{created}\PYG{+w}{ }\PYG{k}{if}\PYG{+w}{ }\PYG{n}{it}\PYG{+w}{ }\PYG{n}{does}
\PYG{+w}{       }\PYG{n}{not}\PYG{+w}{ }\PYG{n+nb}{exist}\PYG{p}{.}\PYG{+w}{ }\PYG{n}{Defaults}\PYG{+w}{ }\PYG{n}{to}\PYG{+w}{ }\PYG{n}{the}\PYG{+w}{ }\PYG{n}{filePath}\PYG{+w}{ }\PYG{k}{if}\PYG{+w}{ }\PYG{n}{not}\PYG{+w}{ }\PYG{n}{provided}\PYG{p}{.}
\PYG{+w}{   }\PYG{n}{diagnosticFlag}\PYG{+w}{ }\PYG{p}{:}\PYG{+w}{ }\PYG{n+nb}{double}\PYG{p}{,}\PYG{+w}{ }\PYG{n+nb}{logical}\PYG{p}{,}\PYG{+w}{ }\PYG{n}{optional}
\PYG{+w}{       }\PYG{n}{If}\PYG{+w}{ }\PYG{n+nb}{set}\PYG{+w}{ }\PYG{n}{to}\PYG{+w}{ }\PYG{l+m+mi}{1}\PYG{p}{,}\PYG{+w}{ }\PYG{n}{the}\PYG{+w}{ }\PYG{k}{function}\PYG{+w}{ }\PYG{n}{displays}\PYG{+w}{ }\PYG{n}{the}\PYG{+w}{ }\PYG{n}{files}\PYG{+w}{ }\PYG{n}{in}\PYG{+w}{ }\PYG{n}{the}\PYG{+w}{ }\PYG{n}{command}\PYG{+w}{ }\PYG{n}{window}\PYG{+w}{ }\PYG{n}{and}\PYG{+w}{ }\PYG{n}{does}
\PYG{+w}{       }\PYG{n}{not}\PYG{+w}{ }\PYG{k}{continue}\PYG{+w}{ }\PYG{n}{processing}\PYG{p}{.}\PYG{+w}{ }\PYG{n}{Defaults}\PYG{+w}{ }\PYG{n}{to}\PYG{+w}{ }\PYG{l+m+mf}{0.}

\PYG{+w}{   }\PYG{n}{Notes}
\PYG{+w}{   }\PYG{o}{\PYGZhy{}}\PYG{o}{\PYGZhy{}}\PYG{o}{\PYGZhy{}}\PYG{o}{\PYGZhy{}}\PYG{o}{\PYGZhy{}}
\PYG{+w}{   }\PYG{n}{The}\PYG{+w}{ }\PYG{k}{function}\PYG{+w}{ }\PYG{n}{adds}\PYG{+w}{ }\PYG{n}{necessary}\PYG{+w}{ }\PYG{n}{paths}\PYG{+w}{ }\PYG{k}{for}\PYG{+w}{ }\PYG{n}{ScanImage}\PYG{+w}{ }\PYG{n}{utilities}\PYG{+w}{ }\PYG{n}{and}\PYG{+w}{ }\PYG{n}{processes}\PYG{+w}{ }\PYG{n}{each}\PYG{+w}{ }\PYG{p}{.}\PYG{n}{tif}
\PYG{+w}{   }\PYG{n}{file}\PYG{+w}{ }\PYG{n}{found}\PYG{+w}{ }\PYG{n}{in}\PYG{+w}{ }\PYG{n}{the}\PYG{+w}{ }\PYG{n}{specified}\PYG{+w}{ }\PYG{n}{directory}\PYG{p}{.}\PYG{+w}{ }\PYG{n}{It}\PYG{+w}{ }\PYG{n}{checks}\PYG{+w}{ }\PYG{k}{if}\PYG{+w}{ }\PYG{n}{the}\PYG{+w}{ }\PYG{n}{directory}\PYG{+w}{ }\PYG{n}{exists}\PYG{p}{,}\PYG{+w}{ }\PYG{n}{handles}
\PYG{+w}{   }\PYG{n}{multiple}\PYG{+w}{ }\PYG{n}{or}\PYG{+w}{ }\PYG{n+nb}{single}\PYG{+w}{ }\PYG{n}{file}\PYG{+w}{ }\PYG{n}{scenarios}\PYG{p}{,}\PYG{+w}{ }\PYG{n}{and}\PYG{+w}{ }\PYG{n}{can}\PYG{+w}{ }\PYG{n}{optionally}\PYG{+w}{ }\PYG{n}{report}\PYG{+w}{ }\PYG{n}{the}\PYG{+w}{ }\PYG{n}{directory}\PYG{o}{\PYGZsq{}}\PYG{n}{s}\PYG{+w}{ }\PYG{n}{contents}
\PYG{+w}{   }\PYG{n}{based}\PYG{+w}{ }\PYG{n}{on}\PYG{+w}{ }\PYG{n}{the}\PYG{+w}{ }\PYG{n}{diagnosticFlag}\PYG{p}{.}

\PYG{+w}{   }\PYG{n}{Each}\PYG{+w}{ }\PYG{n}{file}\PYG{+w}{ }\PYG{n}{processed}\PYG{+w}{ }\PYG{n}{is}\PYG{+w}{ }\PYG{n}{logged}\PYG{p}{,}\PYG{+w}{ }\PYG{n}{assembled}\PYG{+w}{ }\PYG{n}{into}\PYG{+w}{ }\PYG{n}{a}\PYG{+w}{ }4\PYG{n}{D}\PYG{+w}{ }\PYG{n+nb}{volume}\PYG{p}{,}\PYG{+w}{ }\PYG{n}{and}\PYG{+w}{ }\PYG{n}{saved}\PYG{+w}{ }\PYG{n}{in}\PYG{+w}{ }\PYG{n}{a}\PYG{+w}{ }\PYG{n}{specified}
\PYG{+w}{   }\PYG{n}{directory}\PYG{+w}{ }\PYG{n}{as}\PYG{+w}{ }\PYG{n}{a}\PYG{+w}{ }\PYG{p}{.}\PYG{n}{mat}\PYG{+w}{ }\PYG{n}{file}\PYG{+w}{ }\PYG{n}{with}\PYG{+w}{ }\PYG{n}{accompanying}\PYG{+w}{ }\PYG{n}{metadata}\PYG{p}{.}\PYG{+w}{ }\PYG{n}{The}\PYG{+w}{ }\PYG{k}{function}\PYG{+w}{ }\PYG{n}{also}\PYG{+w}{ }\PYG{n}{manages}\PYG{+w}{ }\PYG{n}{errors}
\PYG{+w}{   }\PYG{n}{by}\PYG{+w}{ }\PYG{n}{cleaning}\PYG{+w}{ }\PYG{n}{up}\PYG{+w}{ }\PYG{n}{and}\PYG{+w}{ }\PYG{n}{providing}\PYG{+w}{ }\PYG{n}{detailed}\PYG{+w}{ }\PYG{n+nb}{error}\PYG{+w}{ }\PYG{n}{messages}\PYG{+w}{ }\PYG{k}{if}\PYG{+w}{ }\PYG{n}{something}\PYG{+w}{ }\PYG{n}{goes}\PYG{+w}{ }\PYG{n}{wrong}\PYG{+w}{ }\PYG{n}{during}
\PYG{+w}{   }\PYG{n}{processing}\PYG{p}{.}

\PYG{+w}{   }\PYG{n}{Examples}
\PYG{+w}{   }\PYG{o}{\PYGZhy{}}\PYG{o}{\PYGZhy{}}\PYG{o}{\PYGZhy{}}\PYG{o}{\PYGZhy{}}\PYG{o}{\PYGZhy{}}\PYG{o}{\PYGZhy{}}\PYG{o}{\PYGZhy{}}\PYG{o}{\PYGZhy{}}
\PYG{+w}{   }\PYG{p}{.}\PYG{p}{.}\PYG{+w}{ }\PYG{n}{code}\PYG{o}{\PYGZhy{}}\PYG{n}{block}\PYG{p}{:}\PYG{p}{:}\PYG{+w}{ }\PYG{n}{MATLAB}

\PYG{+w}{         }\PYG{c}{\PYGZpc{} Path to data, path to save data, diagnostic flag}
\PYG{+w}{         }\PYG{n}{convertScanImageTiffToVolume}\PYG{p}{(}\PYG{l+s}{\PYGZsq{}}\PYG{l+s}{C:/data/session1/\PYGZsq{}}\PYG{p}{,}\PYG{+w}{ }\PYG{l+s}{\PYGZsq{}}\PYG{l+s}{C:/processed/\PYGZsq{}}\PYG{p}{,}\PYG{+w}{ }\PYG{l+m+mi}{0}\PYG{p}{)}\PYG{p}{;}
\PYG{+w}{         }\PYG{n}{convertScanImageTiffToVolume}\PYG{p}{(}\PYG{l+s}{\PYGZsq{}}\PYG{l+s}{C:/data/session1/\PYGZsq{}}\PYG{p}{,}\PYG{+w}{ }\PYG{l+s}{\PYGZsq{}}\PYG{l+s}{C:/processed/\PYGZsq{}}\PYG{p}{,}\PYG{+w}{ }\PYG{l+m+mi}{1}\PYG{p}{)}\PYG{p}{;}\PYG{+w}{ }\PYG{c}{\PYGZpc{} just display files}

\PYG{+w}{   }\PYG{n}{See}\PYG{+w}{ }\PYG{n}{also}\PYG{+w}{ }\PYG{n+nb}{fileparts}\PYG{p}{,}\PYG{+w}{ }\PYG{n+nb}{addpath}\PYG{p}{,}\PYG{+w}{ }\PYG{n+nb}{genpath}\PYG{p}{,}\PYG{+w}{ }\PYG{n+nb}{isfolder}\PYG{p}{,}\PYG{+w}{ }\PYG{n+nb}{dir}\PYG{p}{,}\PYG{+w}{ }\PYG{n+nb}{fullfile}\PYG{p}{,}\PYG{+w}{ }\PYG{n+nb}{error}\PYG{p}{,}\PYG{+w}{ }\PYG{n+nb}{regexp}\PYG{p}{,}\PYG{+w}{ }\PYG{n}{savefast}
\end{sphinxVerbatim}


\subsubsection{MATLAB and Python}
\label{\detokenize{user_guide/tips_tricks:matlab-and-python}}
\sphinxAtStartPar
Transitioning data pipelines between MATLAB and Python can be tricky. The two primary reasons for this are the indexing and row/column major array operations.


\paragraph{Indexing}
\label{\detokenize{user_guide/tips_tricks:indexing}}
\sphinxAtStartPar
In modern\sphinxhyphen{}day computer science, most programming languages such as Python, Ruby, PHP, and Java have array indices starting at zero.
A big reason for this is that it provides a clear distinction that ordinal forms (e.g. first, second, third) has a well\sphinxhyphen{}established meaning that the zeroth derivative of a function.

\sphinxAtStartPar
Matlab, like Julia, was created for scientific computing tailored to beginners and thus adopted the more intuitive 1 based indexing.


\paragraph{Row/Column Operations}
\label{\detokenize{user_guide/tips_tricks:row-column-operations}}
\sphinxAtStartPar
In terms of practically transfering data between programming languages, 0 or 1 based indexing can be managed by single \sphinxhref{https://stackoverflow.com/a/7233597/12953787}{enumerating} for loops.


\subsubsection{Number of Cores/Workers}
\label{\detokenize{user_guide/tips_tricks:number-of-cores-workers}}
\sphinxAtStartPar
By default, Matlab creates as many workers as logical CPU cores. On Intel CPUs, the OS reports two logical cores per each physical core due to hyper\sphinxhyphen{}threading, for a total of 4 workers on a dual\sphinxhyphen{}core machine. However, in many situations, hyperthreading does not improve the performance of a program and may even degrade it (I deliberately wish to avoid the heated debate over this: you can find endless discussions about it online and decide for yourself). Coupled with the non\sphinxhyphen{}negligible overhead of starting, coordinating and communicating with twice as many Matlab instances (workers are headless {[}=GUI\sphinxhyphen{}less{]} Matlab processes after all), we reach a conclusion that it may actually be better in many cases to use only as many workers as physical (not logical) cores.
I know the documentation and configuration panel seem to imply that parpool uses the number of physical cores by default, but in my tests I have seen otherwise (namely, logical cores). Maybe this is system\sphinxhyphen{}dependent, and maybe there is a switch somewhere that controls this, I don’t know. I just know that in many cases I found it beneficial to reduce the number of workers to the actual number of physical cores:

\begin{sphinxVerbatim}[commandchars=\\\{\}]
\PYG{n}{p}\PYG{+w}{ }\PYG{p}{=}\PYG{+w}{ }\PYG{n}{parpool}\PYG{p}{;}\PYG{+w}{     }\PYG{c}{\PYGZpc{} NOT RECOMMENDED, CaImAn will very likely run out of resources error}
\PYG{n}{p}\PYG{+w}{ }\PYG{p}{=}\PYG{+w}{ }\PYG{n}{parpool}\PYG{p}{(}\PYG{l+m+mi}{2}\PYG{p}{)}\PYG{p}{;}\PYG{+w}{  }\PYG{c}{\PYGZpc{} use only 2 parallel workers}
\end{sphinxVerbatim}

\sphinxAtStartPar
This can vary greatly across programs and platforms, so you should first ensure the pipeline will run using \textless{}1/2 available cores before increasing the compute demands.
It would of course be better to dynamically retrieve the number of physical cores, rather than hard\sphinxhyphen{}coding a constant value (number of workers) into our program.

\sphinxAtStartPar
We can get this value in Matlab using the undocumented feature(‘numcores’) function:

\begin{sphinxVerbatim}[commandchars=\\\{\}]
\PYG{n}{numCores}\PYG{+w}{ }\PYG{p}{=}\PYG{+w}{ }\PYG{n}{feature}\PYG{p}{(}\PYG{l+s}{\PYGZsq{}}\PYG{l+s}{numcores\PYGZsq{}}\PYG{p}{)}\PYG{p}{;}
\PYG{n}{p}\PYG{+w}{ }\PYG{p}{=}\PYG{+w}{ }\PYG{n}{parpool}\PYG{p}{(}\PYG{n}{numCores}\PYG{p}{)}\PYG{p}{;}
\end{sphinxVerbatim}

\sphinxAtStartPar
Running \sphinxcode{\sphinxupquote{feature(‘numcores’)}} without assigning its output displays some general debugging information:

\begin{sphinxVerbatim}[commandchars=\\\{\}]
\PYG{o}{\PYGZgt{}}\PYG{o}{\PYGZgt{}}\PYG{+w}{ }\PYG{n}{feature}\PYG{p}{(}\PYG{l+s}{\PYGZsq{}}\PYG{l+s}{numcores\PYGZsq{}}\PYG{p}{)}
\PYG{n}{MATLAB}\PYG{+w}{ }\PYG{l+s}{detected:}\PYG{+w}{ }\PYG{l+s}{24}\PYG{+w}{ }\PYG{l+s}{physical}\PYG{+w}{ }\PYG{l+s}{cores.}
\PYG{n}{MATLAB}\PYG{+w}{ }\PYG{l+s}{detected:}\PYG{+w}{ }\PYG{l+s}{32}\PYG{+w}{ }\PYG{l+s}{logical}\PYG{+w}{ }\PYG{l+s}{cores.}
\PYG{n}{MATLAB}\PYG{+w}{ }\PYG{l+s}{was}\PYG{+w}{ }\PYG{l+s}{assigned:}\PYG{+w}{ }\PYG{l+s}{32}\PYG{+w}{ }\PYG{l+s}{logical}\PYG{+w}{ }\PYG{l+s}{cores}\PYG{+w}{ }\PYG{l+s}{by}\PYG{+w}{ }\PYG{l+s}{the}\PYG{+w}{ }\PYG{l+s}{OS.}
\PYG{n}{MATLAB}\PYG{+w}{ }\PYG{l+s}{is}\PYG{+w}{ }\PYG{l+s}{using:}\PYG{+w}{ }\PYG{l+s}{24}\PYG{+w}{ }\PYG{l+s}{logical}\PYG{+w}{ }\PYG{l+s}{cores.}
\PYG{n}{MATLAB}\PYG{+w}{ }\PYG{l+s}{is}\PYG{+w}{ }\PYG{l+s}{not}\PYG{+w}{ }\PYG{l+s}{using}\PYG{+w}{ }\PYG{l+s}{all}\PYG{+w}{ }\PYG{l+s}{logical}\PYG{+w}{ }\PYG{l+s}{cores}\PYG{+w}{ }\PYG{l+s}{because}\PYG{+w}{ }\PYG{l+s}{hyper\PYGZhy{}threading}\PYG{+w}{ }\PYG{l+s}{is}\PYG{+w}{ }\PYG{l+s}{enabled.}

\PYG{n+nb}{ans}\PYG{+w}{ }\PYG{p}{=}

\PYG{+w}{    }\PYG{l+m+mi}{24}
\end{sphinxVerbatim}

\sphinxAtStartPar
You can use this return value to decide how how much of your computers total processing power should be dedicated toward running this pipeline:

\begin{sphinxVerbatim}[commandchars=\\\{\}]
\PYG{o}{\PYGZgt{}}\PYG{o}{\PYGZgt{}}\PYG{+w}{ }\PYG{n}{feature}\PYG{p}{(}\PYG{l+s}{\PYGZsq{}}\PYG{l+s}{numcores\PYGZsq{}}\PYG{p}{)}\PYG{+w}{ }\PYG{o}{\PYGZhy{}}\PYG{+w}{ }\PYG{l+m+mi}{2}\PYG{+w}{ }\PYG{c}{\PYGZpc{} leave 2 cores open for the rest of the system}

\PYG{n+nb}{ans}\PYG{+w}{ }\PYG{p}{=}

\PYG{+w}{    }\PYG{l+m+mi}{23}
\end{sphinxVerbatim}

\sphinxAtStartPar
This specific tip is equally valid for parfor/eval loops and spmd blocks, since both of them use the pool of workers started by parpool.

\sphinxstepscope


\subsection{Troubleshooting}
\label{\detokenize{user_guide/troubleshooting:troubleshooting}}\label{\detokenize{user_guide/troubleshooting::doc}}

\subsubsection{Memory}
\label{\detokenize{user_guide/troubleshooting:memory}}\begin{itemize}
\item {} 
\sphinxAtStartPar
Number of Workers/Cores \textgreater{} 100: There a a known bug in MATLAB R2023a for cases when the number of workers is more than 100.
Refer to the following \sphinxhref{https://www.mathworks.com/support/bugreports/details/2968710.html\textasciigrave{}}{bug report} for a workaround to resolve the issue. Additionally, steps taken in {\hyperref[\detokenize{user_guide/troubleshooting:matlab-server-issues}]{\sphinxcrossref{\DUrole{std,std-ref}{Matlab Server Issues}}}} can help to solve this problem.

\item {} 
\sphinxAtStartPar
Out of Memory during deserialization

\sphinxAtStartPar
Sometimes during motion\sphinxhyphen{}correction, NormCorre will use more memory than it is supposed to. See \textless{}Issue Link\textgreater{}.
If you’re using all of the cores your computing environment has available, that is almost certainly the cause. Decrease
the number of cores as the third input of {\hyperref[\detokenize{api/core:motionCorrectPlane}]{\sphinxcrossref{\sphinxcode{\sphinxupquote{motionCorrectPlane()}}}}}. If this doesn’t correct the issue, it is likely due to
a single 3D\sphinxhyphen{}planar time\sphinxhyphen{}series being too large to fit in memory, in which case you can prevent caiman from processing the image patches
in parallel. Keep in mind this will be noticably slower than the parallel counterpart.

\end{itemize}


\subsubsection{Missing Compiled Binary (Windows)}
\label{\detokenize{user_guide/troubleshooting:missing-compiled-binary-windows}}\begin{itemize}
\item {} 
\sphinxAtStartPar
Typically seen as: \sphinxtitleref{run\_CNMF\_patches} function errors out on Windows.

\end{itemize}

\sphinxAtStartPar
\sphinxstylestrong{Cause:} Likely caused by missing compiled binary for \sphinxtitleref{graph\_conn\_comp\_mex.mexw64 (win)}/ \sphinxtitleref{graph\_conn\_comp\_mex.mexa64 (unix)}

\sphinxAtStartPar
\sphinxstylestrong{Solution:}
1. Compile the binary in MATLAB via the command window:

\begin{sphinxVerbatim}[commandchars=\\\{\}]
\PYG{o}{\PYGZgt{}}\PYG{o}{\PYGZgt{}}\PYG{+w}{ }\PYG{n+nb}{mex}\PYG{+w}{ }\PYG{o}{.\PYGZbs{}}\PYG{n}{CaImAn\PYGZus{}Utilities}\PYG{o}{\PYGZbs{}}\PYG{n}{CaImAn}\PYG{o}{\PYGZhy{}}\PYG{n}{MATLAB}\PYG{o}{\PYGZhy{}}\PYG{n}{master}\PYG{o}{\PYGZbs{}}\PYG{n}{CaImAn}\PYG{o}{\PYGZhy{}}\PYG{n}{MATLAB}\PYG{o}{\PYGZhy{}}\PYG{n}{master}\PYG{o}{\PYGZbs{}}\PYG{n}{utilities}\PYG{o}{\PYGZbs{}}\PYG{n}{graph\PYGZus{}conn\PYGZus{}comp\PYGZus{}mex}\PYG{p}{.}\PYG{n}{cpp}
\PYG{n}{Building}\PYG{+w}{ }\PYG{l+s}{with}\PYG{+w}{ }\PYG{l+s}{\PYGZsq{}MinGW64 Compiler (C++)\PYGZsq{}}\PYG{l+s}{.}
\PYG{n}{MEX}\PYG{+w}{ }\PYG{l+s}{completed}\PYG{+w}{ }\PYG{l+s}{successfully.}
\end{sphinxVerbatim}

\begin{sphinxadmonition}{note}{Note:}
\sphinxAtStartPar
Newest version 0.2.0+ include both precompiled binaries.
\end{sphinxadmonition}


\subsubsection{Matlab Server Issues}
\label{\detokenize{user_guide/troubleshooting:matlab-server-issues}}\phantomsection\label{\detokenize{user_guide/troubleshooting:server-issues}}
\sphinxAtStartPar
These come in many flavors and are mostly \sphinxtitleref{windows} issues due to their background serrvice.

\sphinxAtStartPar
Here is the general fix for all of them:

\sphinxAtStartPar
1. Task Manager:
\sphinxhyphen{} End all MATLAB\sphinxhyphen{}related tasks.
2. Check MATLAB License:
\sphinxhyphen{} Run \sphinxtitleref{license checkout Distrib\_Computing\_Toolbox}.
\sphinxhyphen{} If \sphinxtitleref{Ans=1}, the license is valid.
3. Revert Local Profile:
\sphinxhyphen{} Create a new profile in the cluster manager, set it as default, and delete ‘Processes’.
4. Replace Local Cluster Storage:
\sphinxhyphen{} Find \sphinxtitleref{prefdir} using \sphinxtitleref{\textgreater{}\textgreater{} prefdir} (e.g., \sphinxtitleref{C:Users\%username\%AppDataRoamingMathWorksMATLABR202x}).
\sphinxhyphen{} Delete the \sphinxtitleref{MATLABlocal\_cluster\_jobs} directory one level up.
5. Check for Potentially Conflicting Files:
\sphinxhyphen{} Run \sphinxtitleref{which \sphinxhyphen{}all startup.m}. If not found, it’s not the issue.
\sphinxhyphen{} Run \sphinxtitleref{which \sphinxhyphen{}all pathdef.m}. Ensure it’s located in \sphinxtitleref{C:Program FilesMATLABR2023btoolboxlocalpathdef.m}.
\sphinxhyphen{} Run \sphinxtitleref{which \sphinxhyphen{}all matlabrc.m}. Ensure it’s located in \sphinxtitleref{C:Program FilesMATLABR2023btoolboxlocalmatlabrc.m}.
\begin{quote}
\end{quote}

\sphinxstepscope


\section{API}
\label{\detokenize{api/index:api}}\label{\detokenize{api/index::doc}}
\sphinxAtStartPar
Documentation for each function from docstrings.

\sphinxstepscope


\subsection{Core Functions}
\label{\detokenize{api/core:core-functions}}\label{\detokenize{api/core::doc}}\index{assembleCorrectedROITiff() (built\sphinxhyphen{}in function)@\spxentry{assembleCorrectedROITiff()}\spxextra{built\sphinxhyphen{}in function}}

\begin{fulllineitems}
\phantomsection\label{\detokenize{api/core:assembleCorrectedROITiff}}
\pysigstartsignatures
\pysiglinewithargsret{\sphinxbfcode{\sphinxupquote{assembleCorrectedROITiff}}}{}{}
\pysigstopsignatures
\end{fulllineitems}

\index{convertScanImageTiffToVolume() (built\sphinxhyphen{}in function)@\spxentry{convertScanImageTiffToVolume()}\spxextra{built\sphinxhyphen{}in function}}

\begin{fulllineitems}
\phantomsection\label{\detokenize{api/core:convertScanImageTiffToVolume}}
\pysigstartsignatures
\pysiglinewithargsret{\sphinxbfcode{\sphinxupquote{convertScanImageTiffToVolume}}}{\sphinxparam{data\_path}\sphinxparamcomma \sphinxparam{save\_path}\sphinxparamcomma \sphinxparam{varargin}}{}
\pysigstopsignatures
\sphinxAtStartPar
convertScanImageTiffToVolume Convert ScanImage .tif files into a 4D volume.

\sphinxAtStartPar
Convert raw \sphinxhref{https://www.mbfbioscience.com/products/scanimage/}{ScanImage} multi\sphinxhyphen{}roi .tif files from a single session
into a single 4D volumetric time\sphinxhyphen{}series (x, y, z, t). It’s designed to process files for the
ScanImage Version: 2016 software.
\begin{quote}\begin{description}
\sphinxlineitem{Parameters}\begin{description}
\sphinxlineitem{\sphinxstylestrong{data\_path}}{[}char{]}
\sphinxAtStartPar
The directory containing the raw .tif files. Only raw .tif files from one
session should be in the directory.

\sphinxlineitem{\sphinxstylestrong{save\_path}}{[}char, optional{]}
\sphinxAtStartPar
The directory where processed files will be saved. It is created if it does
not exist. Defaults to the data\_path directory.

\sphinxlineitem{\sphinxstylestrong{group\_path}}{[}string, optional{]}
\sphinxAtStartPar
Group path within the hdf5 file to save the extracted data. Default is
‘/extraction’.

\sphinxlineitem{\sphinxstylestrong{debug\_flag}}{[}double, logical, optional{]}
\sphinxAtStartPar
If set to 1, the function displays the files in the command window and does
not continue processing. Defaults to 0.

\sphinxlineitem{\sphinxstylestrong{overwrite}}{[}logical, optional{]}
\sphinxAtStartPar
Whether to overwrite existing files (default is 1).

\sphinxlineitem{\sphinxstylestrong{fix\_scan\_phase}}{[}logical, optional{]}
\sphinxAtStartPar
Whether to correct for bi\sphinxhyphen{}directional scan artifacts. (default is true).

\sphinxlineitem{\sphinxstylestrong{trim\_pixels}}{[}double, optional{]}
\sphinxAtStartPar
Pixels to trim from left, right,top, bottom of each scanfield before
horizontally concatenating the scanfields within an image. Default is
{[}0 0 0 0{]}.

\sphinxlineitem{\sphinxstylestrong{compression}}{[}double, optional{]}
\sphinxAtStartPar
Compression level for the file (default is 0).

\end{description}

\end{description}\end{quote}
\subsubsection*{Notes}

\sphinxAtStartPar
The function adds necessary paths for ScanImage utilities and processes each .tif
file found in the specified directory. It checks if the directory exists, handles
multiple or single file scenarios, and can optionally report the directory’s contents
based on the debug\_flag.

\sphinxAtStartPar
Each file processed is logged, assembled into a 4D volume, and saved in a specified
directory as a .mat file with accompanying metadata. The function also manages errors
by cleaning up and providing detailed error messages if something goes wrong during
processing.

\sphinxAtStartPar
See also \sphinxcode{\sphinxupquote{FILEPARTS}}, \sphinxcode{\sphinxupquote{ADDPATH}}, \sphinxcode{\sphinxupquote{GENPATH}}, \sphinxcode{\sphinxupquote{ISFOLDER}}, \sphinxcode{\sphinxupquote{DIR}}, \sphinxcode{\sphinxupquote{FULLFILE}}, \sphinxcode{\sphinxupquote{ERROR}}, \sphinxcode{\sphinxupquote{REGEXP}}, \sphinxcode{\sphinxupquote{SAVEFAST}}

\end{fulllineitems}

\index{motionCorrectPlane() (built\sphinxhyphen{}in function)@\spxentry{motionCorrectPlane()}\spxextra{built\sphinxhyphen{}in function}}

\begin{fulllineitems}
\phantomsection\label{\detokenize{api/core:motionCorrectPlane}}
\pysigstartsignatures
\pysiglinewithargsret{\sphinxbfcode{\sphinxupquote{motionCorrectPlane}}}{\sphinxparam{data\_path}\sphinxparamcomma \sphinxparam{save\_path}\sphinxparamcomma \sphinxparam{varargin}}{}
\pysigstopsignatures
\sphinxAtStartPar
MOTIONCORRECTPLANE Perform rigid and non\sphinxhyphen{}rigid motion correction on imaging data.
\begin{quote}\begin{description}
\sphinxlineitem{Parameters}\begin{description}
\sphinxlineitem{\sphinxstylestrong{data\_path}}{[}char{]}
\sphinxAtStartPar
Path to the directory containing the files extracted via convertScanImageTiffToVolume.

\sphinxlineitem{\sphinxstylestrong{save\_path}}{[}char{]}
\sphinxAtStartPar
Path to the directory to save the motion vectors.

\sphinxlineitem{\sphinxstylestrong{data\_input\_group}}{[}string, optional{]}
\sphinxAtStartPar
Group path within the hdf5 file that contains raw data.
Default is ‘registration’.

\sphinxlineitem{\sphinxstylestrong{data\_output\_group}}{[}string, optional{]}
\sphinxAtStartPar
Group path within the hdf5 file to save the registered data.
Default is ‘registration’.

\sphinxlineitem{\sphinxstylestrong{debug\_flag}}{[}double, logical, optional{]}
\sphinxAtStartPar
If set to 1, the function displays the files in the command window and does
not continue processing. Defaults to 0.

\sphinxlineitem{\sphinxstylestrong{overwrite}}{[}logical, optional{]}
\sphinxAtStartPar
Whether to overwrite existing files (default is 1).

\sphinxlineitem{\sphinxstylestrong{num\_cores}}{[}double, integer, positive{]}
\sphinxAtStartPar
Number of cores to use for computation. The value is limited to a maximum
of 24 cores.

\sphinxlineitem{\sphinxstylestrong{start\_plane}}{[}double, integer, positive{]}
\sphinxAtStartPar
The starting plane index for processing.

\sphinxlineitem{\sphinxstylestrong{end\_plane}}{[}double, integer, positive{]}
\sphinxAtStartPar
The ending plane index for processing. Must be greater than or equal to
start\_plane.

\end{description}

\sphinxlineitem{Returns}\begin{description}
\sphinxlineitem{\sphinxstylestrong{shifts}}{[}array{]}
\sphinxAtStartPar
2D motion vectors as single precision.

\end{description}

\end{description}\end{quote}
\subsubsection*{Notes}
\begin{itemize}
\item {} 
\sphinxAtStartPar
Each motion\sphinxhyphen{}corrected plane is saved as a .hdf5 group containing the 2D
shift vectors in x and y

\item {} 
\sphinxAtStartPar
Only .h5 files containing processed volumes should be in the file\_path.

\end{itemize}

\end{fulllineitems}

\index{segmentPlane() (built\sphinxhyphen{}in function)@\spxentry{segmentPlane()}\spxextra{built\sphinxhyphen{}in function}}

\begin{fulllineitems}
\phantomsection\label{\detokenize{api/core:segmentPlane}}
\pysigstartsignatures
\pysiglinewithargsret{\sphinxbfcode{\sphinxupquote{segmentPlane}}}{\sphinxparam{data\_path}\sphinxparamcomma \sphinxparam{save\_path}\sphinxparamcomma \sphinxparam{varargin}}{}
\pysigstopsignatures
\sphinxAtStartPar
SEGMENTPLANE Segment imaging data using CaImAn for motion\sphinxhyphen{}corrected data.

\sphinxAtStartPar
This function applies the CaImAn algorithm to segment neurons from
motion\sphinxhyphen{}corrected, pre\sphinxhyphen{}processed and ROI re\sphinxhyphen{}assembled MAxiMuM data.
The processing is conducted for specified planes, and the results
are saved to disk.
\begin{quote}\begin{description}
\sphinxlineitem{Parameters}\begin{description}
\sphinxlineitem{\sphinxstylestrong{data\_path}}{[}char{]}
\sphinxAtStartPar
Path to the directory containing the files extracted via convertScanImageTiffToVolume.

\sphinxlineitem{\sphinxstylestrong{save\_path}}{[}char{]}
\sphinxAtStartPar
Path to the directory to save the motion vectors.

\sphinxlineitem{\sphinxstylestrong{data\_input\_group}}{[}string, optional{]}
\sphinxAtStartPar
Group path within the hdf5 file that contains raw data.
Default is ‘registration’.

\sphinxlineitem{\sphinxstylestrong{data\_output\_group}}{[}string, optional{]}
\sphinxAtStartPar
Group path within the hdf5 file to save the registered data.
Default is ‘registration’.

\sphinxlineitem{\sphinxstylestrong{debug\_flag}}{[}double, logical, optional{]}
\sphinxAtStartPar
If set to 1, the function displays the files in the command window and does
not continue processing. Defaults to 0.

\sphinxlineitem{\sphinxstylestrong{overwrite}}{[}logical, optional{]}
\sphinxAtStartPar
Whether to overwrite existing files (default is 1).

\sphinxlineitem{\sphinxstylestrong{num\_cores}}{[}double, integer, positive{]}
\sphinxAtStartPar
Number of cores to use for computation. The value is limited to a maximum
of 24 cores.

\sphinxlineitem{\sphinxstylestrong{start\_plane}}{[}double, integer, positive{]}
\sphinxAtStartPar
The starting plane index for processing.

\sphinxlineitem{\sphinxstylestrong{end\_plane}}{[}double, integer, positive{]}
\sphinxAtStartPar
The ending plane index for processing. Must be greater than or equal to
start\_plane.

\end{description}

\sphinxlineitem{Returns}\begin{description}
\sphinxlineitem{None}
\end{description}

\end{description}\end{quote}
\subsubsection*{Notes}
\begin{itemize}
\item {} 
\sphinxAtStartPar
Outputs are saved to disk, including:

\item {} 
\sphinxAtStartPar
T\_keep: neuronal time series {[}Km, T{]} (single)

\item {} 
\sphinxAtStartPar
Ac\_keep: neuronal footprints {[}2*tau+1, 2*tau+1, Km{]} (single)

\item {} 
\sphinxAtStartPar
C\_keep: denoised time series {[}Km, T{]} (single)

\item {} 
\sphinxAtStartPar
Km: number of neurons found (single)

\item {} 
\sphinxAtStartPar
Cn: correlation image {[}x, y{]} (single)

\item {} 
\sphinxAtStartPar
b: background spatial components {[}x*y, 3{]} (single)

\item {} 
\sphinxAtStartPar
f: background temporal components {[}3, T{]} (single)

\item {} 
\sphinxAtStartPar
acx: centroid in x direction for each neuron {[}1, Km{]} (single)

\item {} 
\sphinxAtStartPar
acy: centroid in y direction for each neuron {[}1, Km{]} (single)

\item {} 
\sphinxAtStartPar
acm: sum of component pixels for each neuron {[}1, Km{]} (single)

\item {} 
\sphinxAtStartPar
The function handles large datasets by processing each plane serially.

\item {} 
\sphinxAtStartPar
The segmentation settings are based on the assumption of 9.2e4 neurons/mm\textasciicircum{}3
density in the imaged volume.

\end{itemize}

\sphinxAtStartPar
See also \sphinxcode{\sphinxupquote{ADDPATH}}, \sphinxcode{\sphinxupquote{FULLFILE}}, \sphinxcode{\sphinxupquote{DIR}}, \sphinxcode{\sphinxupquote{LOAD}}, \sphinxcode{\sphinxupquote{SAVEFAST}}

\end{fulllineitems}

\index{collatePlanes() (built\sphinxhyphen{}in function)@\spxentry{collatePlanes()}\spxextra{built\sphinxhyphen{}in function}}

\begin{fulllineitems}
\phantomsection\label{\detokenize{api/core:collatePlanes}}
\pysigstartsignatures
\pysiglinewithargsret{\sphinxbfcode{\sphinxupquote{collatePlanes}}}{\sphinxparam{dataPath}\sphinxparamcomma \sphinxparam{data}\sphinxparamcomma \sphinxparam{metadata}\sphinxparamcomma \sphinxparam{startDepth}}{}
\pysigstopsignatures
\sphinxAtStartPar
COLLATEPLANES Analyzes and processes imaging data by extracting and correcting features across multiple planes.

\sphinxAtStartPar
This function analyzes imaging data from a specified directory, applying
various thresholds and corrections based on metadata. It processes neuron
activity data, handles z\sphinxhyphen{}plane corrections, and outputs figures representing
neuron distributions along with collated data files.

\sphinxAtStartPar
The function expects the directory to contain ‘caiman\_output\_plane\_*.mat’ files
with variables related to neuronal activity, and uses provided metadata for
processing parameters. It adjusts parameters dynamically based on the content
of metadata and filters, merges data across imaging planes, and performs
z\sphinxhyphen{}plane and field curvature corrections.
\begin{quote}\begin{description}
\sphinxlineitem{Parameters}\begin{description}
\sphinxlineitem{\sphinxstylestrong{dataPath}}{[}string{]}
\sphinxAtStartPar
Path to the directory containing the data files for analysis.

\sphinxlineitem{\sphinxstylestrong{data}}{[}string (unused, placeholder for future use){]}
\sphinxAtStartPar
Placeholder parameter for passing data directly if needed.

\sphinxlineitem{\sphinxstylestrong{metadata}}{[}struct{]}
\sphinxAtStartPar
Structure containing metadata for processing. Must include fields:
r\_thr, pixel\_resolution, min\_snr, frame\_rate, fovx, and fovy.

\sphinxlineitem{\sphinxstylestrong{startDepth}}{[}double{]}
\sphinxAtStartPar
The starting depth (z0) from which processing should begin; if not
provided, a dialog will prompt for input.

\end{description}

\sphinxlineitem{Returns}\begin{description}
\sphinxlineitem{None}
\end{description}

\end{description}\end{quote}
\subsubsection*{Notes}
\begin{itemize}
\item {} 
\sphinxAtStartPar
A .mat file with collated and processed imaging data.

\item {} 
\sphinxAtStartPar
Expects ‘three\_neuron\_mean\_offsets.mat’ and ‘pollen\_calibration\_Z\_vs\_N.fig’
within the dataPath for processing.

\item {} 
\sphinxAtStartPar
The function uses parallel processing for some calculations to improve
performance.

\end{itemize}
\subsubsection*{Examples}
\begin{description}
\sphinxlineitem{collatePlanes(‘C:/data/images/’, ‘’, struct(‘r\_thr’:0.4, ‘pixel\_resolution’:2, ‘min\_snr’:1.5, ‘frame\_rate’:9.61, ‘fovx’:1200, ‘fovy’:1164), 100);}
\sphinxAtStartPar
This example processes data from ‘C:/data/images/’, starting at a depth of 100 microns,
with specified metadata parameters.

\end{description}

\sphinxAtStartPar
See also \sphinxcode{\sphinxupquote{LOAD}}, \sphinxcode{\sphinxupquote{INPUTDLG}}, \sphinxcode{\sphinxupquote{STRUCT}}, \sphinxcode{\sphinxupquote{FULLFILE}}, \sphinxcode{\sphinxupquote{EXIST}}

\end{fulllineitems}

\index{calculateZOffset() (built\sphinxhyphen{}in function)@\spxentry{calculateZOffset()}\spxextra{built\sphinxhyphen{}in function}}

\begin{fulllineitems}
\phantomsection\label{\detokenize{api/core:calculateZOffset}}
\pysigstartsignatures
\pysiglinewithargsret{\sphinxbfcode{\sphinxupquote{calculateZOffset}}}{\sphinxparam{datapath}\sphinxparamcomma \sphinxparam{metadata}\sphinxparamcomma \sphinxparam{startPlane}\sphinxparamcomma \sphinxparam{endPlane}\sphinxparamcomma \sphinxparam{numFeatures}}{}
\pysigstopsignatures
\sphinxAtStartPar
CALCULATEZOFFSET Calculates Z\sphinxhyphen{}axis offsets between consecutive image planes by cross\sphinxhyphen{}correlation.

\sphinxAtStartPar
This function loads image data from specified planes, identifies features
in each plane, and calculates the offset in pixels between these features
across consecutive planes. The function maximizes cross\sphinxhyphen{}correlation on regions
around identified features to determine the best match and thus the offset.
\begin{quote}\begin{description}
\sphinxlineitem{Parameters}\begin{description}
\sphinxlineitem{\sphinxstylestrong{datapath}}{[}string{]}
\sphinxAtStartPar
Path to the directory containing the image data and calibration files.
The function expects to find ‘pollen\_sample\_xy\_calibration.mat’ in this directory along with each caiman\_output\_plane\_N.

\sphinxlineitem{\sphinxstylestrong{metadata}}{[}struct{]}
\sphinxAtStartPar
Structure containing metadata for the image data. Expected to have at
least the ‘pixel\_resolution’ field which is used to scale distances.

\sphinxlineitem{\sphinxstylestrong{startPlane}}{[}int{]}
\sphinxAtStartPar
The starting plane index from which to begin processing.

\sphinxlineitem{\sphinxstylestrong{endPlane}}{[}int{]}
\sphinxAtStartPar
The ending plane index at which to stop processing. The function
calculates offsets from startPlane to endPlane, inclusive.

\sphinxlineitem{\sphinxstylestrong{numFeatures}}{[}int{]}
\sphinxAtStartPar
The number of features to identify and use in each plane for
calculating offsets.

\end{description}

\sphinxlineitem{Returns}\begin{description}
\sphinxlineitem{\sphinxstylestrong{offsets}}{[}Nx2 array{]}
\sphinxAtStartPar
An array of offsets between consecutive planes, where N is the number
of planes processed. Each row corresponds to a plane, and the two columns
represent the calculated offset in pixels along the x and y directions,
respectively.

\end{description}

\end{description}\end{quote}
\subsubsection*{Notes}
\begin{itemize}
\item {} 
\sphinxAtStartPar
This function requires calibration data in input datapath:
\sphinxhyphen{} pollen\_sample\_xy\_calibration.mat

\item {} 
\sphinxAtStartPar
The function uses MATLAB’s \sphinxtitleref{ginput} function for manual feature selection
on the images. It expects the user to manually select the corresponding
points on each plane.

\item {} 
\sphinxAtStartPar
The function assumes that the consecutive images will have some overlap
and that features will be manually identifiable and trackable across planes.

\end{itemize}

\end{fulllineitems}


\sphinxstepscope


\subsection{Utils}
\label{\detokenize{api/utils:utils}}\label{\detokenize{api/utils::doc}}\index{get\_metadata() (built\sphinxhyphen{}in function)@\spxentry{get\_metadata()}\spxextra{built\sphinxhyphen{}in function}}

\begin{fulllineitems}
\phantomsection\label{\detokenize{api/utils:get_metadata}}
\pysigstartsignatures
\pysiglinewithargsret{\sphinxbfcode{\sphinxupquote{get\_metadata}}}{\sphinxparam{filename}}{}
\pysigstopsignatures
\sphinxAtStartPar
GET\_METADATA Extract metadata quickly from a ScanImage TIFF file.

\sphinxAtStartPar
Read and parse Tiff metadata stored in the .tiff header
and ScanImage metadata stored in the ‘Artist’ tag which contains strip sizes/locations and scanning configuration
details in a JSON format.
\begin{quote}\begin{description}
\sphinxlineitem{Parameters}\begin{description}
\sphinxlineitem{\sphinxstylestrong{filename}}{[}char{]}
\sphinxAtStartPar
The full path to the TIFF file from which metadata will be extracted.

\end{description}

\sphinxlineitem{Returns}\begin{description}
\sphinxlineitem{\sphinxstylestrong{metadata\_out}}{[}struct{]}
\sphinxAtStartPar
A struct containing metadata such as center and size of the scan field,
pixel resolution, image dimensions, number of frames, frame rate, and
additional strip data extracted from the TIFF file.

\end{description}

\end{description}\end{quote}
\subsubsection*{Examples}

\sphinxAtStartPar
metadata = get\_metadata(“path/to/file.tif”);

\end{fulllineitems}

\index{combinePlanes() (built\sphinxhyphen{}in function)@\spxentry{combinePlanes()}\spxextra{built\sphinxhyphen{}in function}}

\begin{fulllineitems}
\phantomsection\label{\detokenize{api/utils:combinePlanes}}
\pysigstartsignatures
\pysiglinewithargsret{\sphinxbfcode{\sphinxupquote{combinePlanes}}}{\sphinxparam{h5path}\sphinxparamcomma \sphinxparam{plane}}{}
\pysigstopsignatures
\sphinxAtStartPar
Load metadata

\end{fulllineitems}

\index{planeToMovie() (built\sphinxhyphen{}in function)@\spxentry{planeToMovie()}\spxextra{built\sphinxhyphen{}in function}}

\begin{fulllineitems}
\phantomsection\label{\detokenize{api/utils:planeToMovie}}
\pysigstartsignatures
\pysiglinewithargsret{\sphinxbfcode{\sphinxupquote{planeToMovie}}}{\sphinxparam{data}\sphinxparamcomma \sphinxparam{filename}\sphinxparamcomma \sphinxparam{x}\sphinxparamcomma \sphinxparam{y}\sphinxparamcomma \sphinxparam{frameRate}\sphinxparamcomma \sphinxparam{avgs}\sphinxparamcomma \sphinxparam{zoom}\sphinxparamcomma \sphinxparam{decenter}\sphinxparamcomma \sphinxparam{buffer}\sphinxparamcomma \sphinxparam{crf}\sphinxparamcomma \sphinxparam{transcode\_flag}\sphinxparamcomma \sphinxparam{scale\_flag}}{}
\pysigstopsignatures
\sphinxAtStartPar
PLANETOMOVIE Generate a movie from image data.

\sphinxAtStartPar
This function processes a 3D array of image data to create a video file,
applying optional zooming, cropping, and color scaling. The final video is
saved in AVI format and can be transcoded to MP4.
\begin{description}
\sphinxlineitem{Inputs:}
\sphinxAtStartPar
data \sphinxhyphen{} 3D matrix of image data.
filename \sphinxhyphen{} Name of the output video file.
x \sphinxhyphen{} Horizontal coordinates.
y \sphinxhyphen{} Vertical coordinates.
frameRate \sphinxhyphen{} Frame rate of the output video.
avgs \sphinxhyphen{} Number of frames to average.
zoom \sphinxhyphen{} Zoom factors. (not implemented)
decenter \sphinxhyphen{} Decentering offsets.
crf \sphinxhyphen{} Constant Rate Factor for video quality.

\end{description}

\end{fulllineitems}

\index{planeToH5() (built\sphinxhyphen{}in function)@\spxentry{planeToH5()}\spxextra{built\sphinxhyphen{}in function}}

\begin{fulllineitems}
\phantomsection\label{\detokenize{api/utils:planeToH5}}
\pysigstartsignatures
\pysiglinewithargsret{\sphinxbfcode{\sphinxupquote{planeToH5}}}{\sphinxparam{frames}\sphinxparamcomma \sphinxparam{folder}\sphinxparamcomma \sphinxparam{filename}\sphinxparamcomma \sphinxparam{metadata}\sphinxparamcomma \sphinxparam{nvargs}}{}
\pysigstopsignatures
\sphinxAtStartPar
Save 3D array of image frames to an HDF5 file in various organizational schemes.
\begin{quote}\begin{description}
\sphinxlineitem{Parameters}\begin{description}
\sphinxlineitem{\sphinxstylestrong{frames}}{[}uint16 array{]}
\sphinxAtStartPar
3D time\sphinxhyphen{}series of image data for a single recording plane.
Each slice along the third dimension represents one image frame (timepoint).

\sphinxlineitem{\sphinxstylestrong{folder}}{[}string{]}
\sphinxAtStartPar
Directory path where the HDF5 file will be saved. Must already exist.

\sphinxlineitem{\sphinxstylestrong{filename}}{[}string{]}
\sphinxAtStartPar
Base name of the HDF5 file to create or append to. Automatically appends “.h5” if not included.

\sphinxlineitem{\sphinxstylestrong{metadata}}{[}struct{]}
\sphinxAtStartPar
Structure containing metadata to be saved as attributes in the HDF5 file.

\sphinxlineitem{\sphinxstylestrong{nvargs}}{[}struct (optional){]}\begin{description}
\sphinxlineitem{Structure containing named variables:}\begin{description}
\sphinxlineitem{dataset}{[}string{]}
\sphinxAtStartPar
Name of the dataset within the HDF5 file.

\sphinxlineitem{groupPath}{[}string{]}
\sphinxAtStartPar
HDF5 internal path (group) where the dataset is stored.

\sphinxlineitem{fileMode}{[}string{]}
\sphinxAtStartPar
Can be ‘separate’ (default), ‘singleGroup’, or ‘multiGroup’.

\sphinxlineitem{chunksize}{[}double array{]}
\sphinxAtStartPar
Size of data chunks for HDF5 storage.

\sphinxlineitem{compression}{[}double{]}
\sphinxAtStartPar
Compression level for the data.

\end{description}

\end{description}

\end{description}

\end{description}\end{quote}
\subsubsection*{Notes}
\begin{description}
\sphinxlineitem{The ‘fileMode’ determines how data is organized:}
\sphinxAtStartPar
‘separate’ : Each plane in a separate file.
‘singleGroup’ : All planes in the same group but different datasets.
‘multiGroup’ : All planes in different groups within the same file.

\end{description}

\end{fulllineitems}

\index{planeToTiff() (built\sphinxhyphen{}in function)@\spxentry{planeToTiff()}\spxextra{built\sphinxhyphen{}in function}}

\begin{fulllineitems}
\phantomsection\label{\detokenize{api/utils:planeToTiff}}
\pysigstartsignatures
\pysiglinewithargsret{\sphinxbfcode{\sphinxupquote{planeToTiff}}}{\sphinxparam{varargin}}{}
\pysigstopsignatures
\sphinxAtStartPar
PLANETOTIFF Write image to tif file with specified datatype.
PLANETOTIFF(IMGDATA,HEADER,IMFILE,DATATYPE) exports IMGDATA with HEADER
to TIF file named IMFILE. HEADER is usally obtained by IMFINFO from
original image file, and it can also be left empty. String DATATYPE
specifies data type for the export. Supported data types include
logical, uint8, int8, uint16, int16, uint32, int32, uint64, int64,
single and double.

\sphinxAtStartPar
PLANETOTIFF(IMGDATA,HEADER,IMFILE,DATATYPE,TAG NAME1,TAG VALUE1,TAG NAME2,
TAG VALUE2,…) writes with specified Matlab supported TIF tag values.
These new tag values overide those already defined in HEADER.
\begin{description}
\sphinxlineitem{Note 1:}
\sphinxAtStartPar
to avoid errors such as ‘??? Error using ==\textgreater{} tifflib The value for
MaxSampleValue must be …’, overide tag MaxSampleValue by Matlab
supported values. Or simply remove the tag from HEADER.

\sphinxlineitem{Note 2:}
\sphinxAtStartPar
Overwriting of the existing image files is not checked. Be cautious
with the export image file name.

\sphinxlineitem{Example 1:}
\sphinxAtStartPar
imgdata = imread(‘ngc6543a.jpg’);
header  = imfinfo(‘ngc6543a.jpg’);
imwrite2tif(imgdata,header,’new\_ngc6543a.tif’,’uint8’);

\sphinxlineitem{Example 2:}
\sphinxAtStartPar
imgdata = imread(‘mri.tif’);
imwrite2tif(imgdata,{[}{]},’new\_mri.tif’,’int32’,’Copyright’,’MRI’,
‘Compression’,1);

\end{description}

\sphinxAtStartPar
More information can be found by searching for ‘Exporting Image Data
and Metadata to TIFF Files’ in Matlab Help.
Zhang Jiang
\$Revision: 1.0 \$  \$Date: 2011/02/23 \$

\end{fulllineitems}

\index{writeMatToTiffStack() (built\sphinxhyphen{}in function)@\spxentry{writeMatToTiffStack()}\spxextra{built\sphinxhyphen{}in function}}

\begin{fulllineitems}
\phantomsection\label{\detokenize{api/utils:writeMatToTiffStack}}
\pysigstartsignatures
\pysiglinewithargsret{\sphinxbfcode{\sphinxupquote{writeMatToTiffStack}}}{\sphinxparam{files}\sphinxparamcomma \sphinxparam{savePath}\sphinxparamcomma \sphinxparam{numFrames}}{}
\pysigstopsignatures
\sphinxAtStartPar
WRITEMATTOTIFF Convert a .mat file into a stack.
\begin{quote}\begin{description}
\sphinxlineitem{Parameters}\begin{description}
\sphinxlineitem{\sphinxstylestrong{files}}{[}struct{]}
\sphinxAtStartPar
List of paths to motion\sphinxhyphen{}corrected movies containing \_plane\_ in the filename.

\sphinxlineitem{\sphinxstylestrong{savePath}}{[}str, optional{]}
\sphinxAtStartPar
Path to save the output files. Default is the current directory.

\sphinxlineitem{\sphinxstylestrong{numFrames}}{[}int, optional{]}
\sphinxAtStartPar
Number of frames to save. Default is 2000.

\end{description}

\end{description}\end{quote}

\end{fulllineitems}

\index{translateFrames() (built\sphinxhyphen{}in function)@\spxentry{translateFrames()}\spxextra{built\sphinxhyphen{}in function}}

\begin{fulllineitems}
\phantomsection\label{\detokenize{api/utils:translateFrames}}
\pysigstartsignatures
\pysiglinewithargsret{\sphinxbfcode{\sphinxupquote{translateFrames}}}{\sphinxparam{Y}\sphinxparamcomma \sphinxparam{shifts\_2D}}{}
\pysigstopsignatures
\sphinxAtStartPar
TRANSLATEFRAMES Translate image frames based on provided translation vectors.

\sphinxAtStartPar
This function applies 2D translations to an image time series based on
a series of translation vectors, one per frame. Each frame is translated
independently, and the result is returned as a 3D stack of
(Height x Width x num\_frames) translated frames.

\end{fulllineitems}



\section{Indices and tables}
\label{\detokenize{index:indices-and-tables}}\begin{itemize}
\item {} 
\sphinxAtStartPar
\DUrole{xref,std,std-ref}{genindex}

\item {} 
\sphinxAtStartPar
\DUrole{xref,std,std-ref}{modindex}

\item {} 
\sphinxAtStartPar
\DUrole{xref,std,std-ref}{search}

\end{itemize}



\renewcommand{\indexname}{Index}
\printindex
\end{document}